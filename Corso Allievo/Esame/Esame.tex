\documentclass[a4paper, 12pt]{article}
\usepackage[italian]{babel}
\usepackage[utf8]{inputenc}
\usepackage{amsthm}
\usepackage{vhistory}
\usepackage{hyperref}

\title{Esame: Prova Pratica 08/06/2016}
\author{
Biselli Fabio
%\and
%Burgazzi Elisa 
%\and
%Cappellini Chiara
\and
Do Carmo Luciana
\and
Giovanelli Enrica
\and
Longhi Corrado
%\and
%Mazzari Francesco
%\and
%Monfasani Andrea
%\and
%Nedeljkovic Jovana
%\and
%Schenardi Simone
%\and
%Tommaso Cardinali
%\and
%Valentino Martina
}
\date {}

\frenchspacing

\theoremstyle{remark}
\newtheorem*{nota}{Nota}

\begin{document}

\maketitle

\section*{Programma Attacco-Difesa}
L'allenamento si svolge in $40$ minuti, ogni allenatore dirige una parte di $10$ minuti consecutivi
suddivisi come segue:
\begin{itemize}
\item [0--10] Luciana, attacco piedi a terra:
  \begin{itemize}
  \item[-]Skip basso (con e senza palla);
  \item[-]Elastici (braccia);
  \item[-]Lancio della pallina sopra elastico con piedi a terra (senza e con caricamento).
  \end{itemize}
\item [10--20] Fabio, attacco in stacco:
  \begin{itemize}
  \item[-]Lancio della palla e presa in salto da fermo (lavoro singolo);
  \item[-]Lancio a coppie con rincorsa e presa (a rete);
  \item[-]Lancio del tecnico con attacco (a rete).
  \end{itemize}
\item [20--30] Enrica, difesa a coppie:
  \begin{itemize}
  \item[-]presa della palla bassa;
  \item[-]balzello e appoggio al compagno;
  \item[-]difesa con variazione d'attacco (palleggio = pallonetto, altrimenti attacco normale).
  \end{itemize}
\item [30--40] Corrado, attacco e difesa in gruppi:
  \begin{itemize}
  \item[-]Tre in difesa con i cinesini e tre in attacco, un palleggiatore e uno che appoggia,
  free-ball dal tecnico, tutti fanno la rincorsa (chiamando i passi: "sinistro-destro-sinistro") il
  palleggiatore decide a chi dare la palla (2-3 minuti per gruppo).
  \end{itemize}
\end{itemize}


\end{document}