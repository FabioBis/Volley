\chapter{Introduzione allo Scouting}

Lo scouting è l'azione di rilevamento dei i dati generanti da una partita o da un allenamento.
I dati servono da supporto alla creazione delle statistiche e quindi alle analisi successive, queste confermano o smentiscono le “sensazioni dell’allenatore”, fondate sulla sua
esperienza ed abilità.
L’atleta informato risulta essere più motivato, lo scout aiuta ad eliminare la cultura dell’alibi.

La rilevazione e l'analisi dei dati devono rispondere ai principi dell'oggettività e dell'obiettività.
Il giocatore deve essere valutato per il suo reale rendimento senza condizionamenti emotivi (simpatia o antipatia), senza condizionamenti legati alla spettacolarità di certe azioni.
Lo souting deve essere basato su criteri di valutazione attentamente studiati e codificati dall'allenatore, a cui lo scout-man deve fare fede.

\subsubsection{A cosa serve lo scout?}
Conoscere le caratteristiche dalla proprio squadra;
Conoscere le caratteristiche delle squadre avversarie;
Ottenere dati precisi e misurabili (oggettivi);
Comunicare ed analizzare i dati alla squadra.

\section{Come si effettua lo scout?}
L'allenatore definisce a priori, per ogni fondamentale, i criteri di valutazione in base alle proprie esigenze, ogni fondamentale può avere un numero variabile di valutazioni.
Lo scouting può essere fatto in tempo reale (durante la partita o allenamento), in qual caso si rende necessaria la figura di scout-man, oppure mediante la registrazione della partita.
Nel primo caso, l'allenatore può usufruire dei dati in tempo reale ed eventualmente intervenire adeguatamente, nel secondo i dati saranno disponibili solo per una seconda analisi.
Di seguito viene definito un sistema basato su 5 criteri di valutazione (simboli o gradi di performance), che risulta tra i più utilizzati. Nulla vieta di variarne i criteri di valutazione o di limitare i fondamentali.

I simboli qui sotto riportati sono i seguenti: $\sharp$ (doppio positivo), $+$ (positivo),
$/$ (slash è un simbolo speciale), $-$ (negativo), $--$ (doppio negativo). A volte viene utilizzato
un altro simbolo speciale ($!$).

\subsubsection{Battuta}
\begin{enumerate}
\item[$\sharp$] Ace, 0, 1 o 2 tocchi dell'avversario;
\item[$+$] Battuta positiva, il pallone ritorna direttamente nel campo della squadra al servizio o la squadra che riceve non può attaccare;
\item[$/$] Battuta “slash”, la squadra che riceve ha un attacco obbligato: palla oltre i 4 metri (gioco obbligato sulle bande) o palla troppo attaccata a rete (gioco obbligato in primo tempo);
\item[$-$] Battuta negativa, la squadra che riceve ha tutte le opzioni d'attacco;
\item[$--$] Errore.
\end{enumerate}

\subsubsection{Ricezione}
\begin{enumerate}
\item[$\sharp$] Ricezione perfetta, il palleggiatore ha la palla entro un raggio di al massimo 1,5 mt ad un altezza adeguata per avere tutte le opzioni d'attacco;
\item[$+$] Ricezione positiva, il palleggiatore ha la palla ad un raggio di poco più di 1/1,5 mt (massimo 2,5) ad un altezza adeguata senza avere il gioco al centro;
\item[$/$] Ricezione “slash”, consente un solo tipo di alzata, oltre i 4 metri o troppo attaccata a rete;
\item[$-$] Ricezione negativa, non consente l'attacco (free-ball);
\item[$--$] Errore, ace subito (0, 1 o 2 tocchi);
\end{enumerate}

\subsubsection{Attacco}
\begin{enumerate}
\item[$\sharp$] Attacco punto;
\item[$+$] Attacco positivo, la squadra che difende non può contrattaccare (compreso il muro rigiocato dalla squadra che ha attaccato);
\item[$/$] Attacco “slash”, la squadra che difende riesce a ricostruire;
\item[$-$] Attacco negativo, la squadra che difende ha tutte le opzioni d'attacco o ha ottenuto il punto a muro;
\item[$--$] Errore diretto;
\end{enumerate}

Spesso si registrano anche le direzioni degli attacchi mediante appositi grafici.

\subsubsection{Muro}
\begin{enumerate}
\item[$\sharp$] Muro punto;
\item[$+$] Muro passivo, favorisce la squadra che difende che può contrattacare agevolmente, oppure torna nel campo della squadra che ha attaccato ma che non può ricostruire un attacco (free-ball);
\item[$/$] Muro ininfluente, la squadra che ha attaccato può rigiocare agevolmente;
\item[$-$] Mani-out;
\item[$--$] Invasione a rete;
\end{enumerate}

\subsubsection{Difesa}
\begin{enumerate}
\item[$\sharp$] Difesa difficile, acrobatica, lontana o con un gesto tecnico evidente;
\item[$+$] Difesa positiva, consente la ricostruzione di un'azione;
\item[$/$] Copertura dell'attacco;
\item[$-$] Difesa negativa, non consente la ricostruzione;
\item[$--$] Difesa non tenuta (tocco errato) o errore di posizione;
\end{enumerate}


\subsection{Formule}
In questa sezione vengono definite le sole tre formule date a lezione e che possono essere richieste durante l'esame, queste riguardano la positività in attacco ed in ricezione e l'efficienza in attacco. Si definiscono per ogni giocatore le seguenti quantità: $A_{\sharp}$ il numero degli attacchi punto, $A_{-}$ il numero degli attacchi negativi, $A_{--}$ il numero degli errori in attacco, $A_{T}$ il numero totale degli attacchi, $R_{\sharp}$ il numero di ricezioni perfette, $R_{+}$ il numero delle ricezioni positive ed $R_{T}$ il numero totale delle ricezioni.

\begin{flushleft}
 \[
\textbf{Efficienza in Attacco:} \quad
 \frac{(A_{\sharp} - A_{-} - A_{--})}{A_{T}} \cdot 100
 \]
\end{flushleft}
 
\begin{flushleft}
\[
\textbf{Positività in Attacco:} \quad
 \frac{A_{\sharp}}{A_{T}} \cdot 100
\]
\end{flushleft}

\begin{center}
\[
\textbf{Positività in Ricezione:} \quad
 \frac{(R_{\sharp} + R_{+})}{R_{T}} \cdot 100
\]
\end{center}
