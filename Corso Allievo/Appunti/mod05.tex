\chapter{Le caratteristiche dei movimenti generali e specifici della pallavolo}

La Pallavolo è uno sport di situazione caratterizzato da movimenti aciclici (che non si ripetono in modo sistematico nel tempo), non stereotipati (l'ambiente esterno influisce su di essi) e tattici. \`E uno sport “open skills”, ovvero uno sport in cui le abilità sono messe in evidenza da un ambiente variabile e poco predicibile (influenzati da eventi esterni: gli avversari). Nella Pallavolo non c'è possesso palla.

\section{Metodologia}
Si può classificare l'esercizio (il mezzo) in vario modo:
\begin{itemize}
\item[-]Analitico, sintetico, globale;
\item[-]Ciclico: il medesimo gesto tecnico ripetuto;
\item[-]Guidato: si comunica al giocatore la variazione;
\item[-]Situazionale: si simulano situazioni di gioco (poca prevedibilità).
\end{itemize}


\section{Programmazione}
La programmazione dipende principalmente dal dalla categoria che si deve allenare. Nel minivolley fino all'under 13 ci si basa sulla motricità e sul gioco della pallavolo, mettendo le basi. Nell'under 14--18 si lavora principalmente sulla tecnica.

\section{Prestazione sportiva}
Si basa sulle capacità coordinative, condizionali e di elaborazione tattica. Vengono sviluppate attraverso la metodologia e la programmazione.

\subsection{Capacità Coordinative}
Le capacità coordinative sono le seguenti:
\begin{itemize}
\item[-]\emph{Apprendimento} del movimento;
\item[-]\emph{Controllo} del movimento;
\item[-]\emph{Adattamento} e \emph{trasformazione} del movimento;
\item[-]Tutte le \emph{manifestazioni coordinative secondarie}.
\end{itemize}

\subsection{Capacità Condizionali}
\begin{itemize}
\item[-]\emph{Forza};
\item[-]\emph{Rapidità};
\item[-]\emph{Resistenza}.
\end{itemize}
Sono caratteristiche del movimento della pallavolo.

%Accelerazione e Decelerazione

\subsection{Complessità dei sistemi}
Controllo delle posture, i processi mentali ed anticipatori analizzati gli stimoli. Intercettazione (tocco della palla) e cambio direzione e dell'altezza (variazione di traiettoria della palla). Lateralità (emisfero destro/sinistro).

\subsection{Dominanza}
Un emisfero comanda le dominanze e non c'è un uso simmetrico del corpo
Nella pallavolo ci sono movimenti/fondamentali simmetrici e non, in quelli simmetrici si hanno meno problemi.

\subsubsection{Traiettorie}
Fondamentale imparare a leggerle, già dal minivolley.

