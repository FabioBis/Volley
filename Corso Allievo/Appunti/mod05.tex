\chapter{Le caratteristiche dei movimenti generali e specifici}

La Pallavolo è uno sport di situazione caratterizzato da movimenti aciclici (che
non si ripetono in modo sistematico nel tempo), non stereotipati (l'ambiente
esterno influisce su di essi) e tattici. \`E uno sport “open skills”, ovvero uno
sport in cui le abilità sono messe in evidenza da un ambiente variabile e poco
predicibile (influenzati da eventi esterni: gli avversari). Nella Pallavolo non
c'è possesso palla.

\begin{libro}
Per disciplina \emph{open skills} si intende una disciplina caratterizzata da
\emph{abilità motorie aperte}, ossia costantemente utilizzate in risposta
alla variabilità situazionale.
\end{libro}

\subsubsection{I Fondamentali}
I vari comportamenti sequenziali che i giocatori attuano durante le azioni
di gioco rappresentano i \emph{fondamentali}, essi sono:
\begin{itemize}
\item[-]Ricezione (azione di cambiopalla);
\item[-]Alzata (azione di cambiopalla);
\item[-]Attacco e copertura (azione di cambiopalla);
\item[-]Servizio (azione di contrattacco);
\item[-]Muro (azione di contrattacco);
\item[-]Difesa (azione di contrattacco);
\item[-]Ricostruzione e attacco con copertura (azione di contrattacco).
\end{itemize}

\subsubsection{Tecniche di Base}
\begin{itemize}
\item[-]Palleggio;
\item[-]Bagher;
\item[-]Battuta float piedi a terra;
\item[-]Schiacciata;
\item[-]Alcune tecniche di spostamento.
\end{itemize}



\section{Metodologia}
Si può classificare l'esercizio (il mezzo) in vario modo:
\begin{itemize}
\item[-]Analitico, sintetico, globale;
\item[-]Ciclico: il medesimo gesto tecnico ripetuto;
\item[-]Guidato: si comunica al giocatore la variazione;
\item[-]Situazionale: si simulano situazioni di gioco (poca prevedibilità).
\end{itemize}


\section{Programmazione}
La programmazione dipende principalmente dal dalla categoria che si deve
allenare. Nel minivolley fino all'under 13 ci si basa sulla motricità e sul
gioco della pallavolo, mettendo le basi. Nell'under 14--18 si lavora
principalmente sulla tecnica.

\section{Prestazione sportiva}
Si basa sulle capacità coordinative, condizionali e di elaborazione tattica.
Queste vengono sviluppate attraverso la metodologia e la programmazione
dell'allenamento sportivo mediante il principio di adattamento.

\subsection{Capacità Coordinative}
Le capacità coordinative, deputate principalmente all'apprendimento, al 
controllo e alla regolazione del movimento, sono le seguenti:
\begin{itemize}
\item[-]\emph{Apprendimento} del movimento;
\item[-]\emph{Controllo} del movimento;
\item[-]\emph{Adattamento} e \emph{trasformazione} del movimento;
\item[-]Tutte le \emph{manifestazioni coordinative secondarie}.
\end{itemize}

\subsection{Capacità Condizionali}
Sono caratteristiche del movimento della pallavolo, fanno riferimento a 
tutti i meccanismi energetici, strutturali, fisiologici e funzionali del
movimento (capacità organico--muscolari), queste sono:
\begin{itemize}
\item[-]\emph{Forza}: definita nel capitolo \ref{fisico};
\item[-]\emph{Rapidità}: la capacità di reagire ad uno stimolo con la massima
velocità possibile;
\item[-]\emph{Resistenza}:  la capacità di mantenere una
 determinata prestazione, esiste quella generale (aerobica, marginale nella
 pallavolo) e quella speciale (resistenza specifica di gara).
\end{itemize}


\section{Complessità dei sistemi}
Controllo delle posture, i processi mentali ed anticipatori analizzati gli
stimoli. Intercettazione (tocco della palla) e cambio direzione e dell'altezza
(variazione di traiettoria della palla). Lateralità (emisfero destro/sinistro).

\subsection{Dominanza}
Un emisfero comanda le dominanze e non sempre c'è un uso simmetrico del corpo.
La \emph{lateralizzazione} comincia ad essere netta dopo i $5$-$6$ anni e
non procede con la stessa velocità su tutti gli organi e le strutture. Ad
esempio i piedi e gli occhi subiscono un ritardo rispetto alla mano.

Risulta importante (nel minivolley) identificare le dominanze di ogni individuo
ed assecondarle, cercare di correggerle può dare disturbi alla lettura, alla
motricità, all'orientamento, etc.

Nella pallavolo ci sono movimenti/fondamentali simmetrici e non, in quelli
simmetrici si hanno meno problemi.

\subsection{Accelerazione e Decelerazione}
Nella pallavolo il concetto di velocità deve essere correlato a quello di
decelerazione. Il sistema accelerazione/decelerazione (enfatizzato
nell'intercettazione della palla) coinvolge altri sistemi: posture statiche e
dinamiche, processi di elaborazione e anticipazione degli stimoli ambientali,
lateralità e modulazione e frequenza del ritmo dei movimenti.

\subsection{Traiettorie}
La conoscenza delle traiettorie avviene per \emph{prove ed errori} e la
misura del processo di apprendimento è la \emph{stabilità} dell'efficacia
motoria, ovvero non ci sono oscillazioni nel numero di esecuzioni corrette e
nel tipo di errore negli esercizi.
Risulta fondamentale imparare a leggere le traiettorie già dal minivolley.

Si evidenziano due fattori condizionanti per la lettura delle traiettorie
ed i movimenti per l'intercettazione: i \emph{posizionamenti} ed il
\emph{concetto di tempo tecnico}.

\begin{libro}
Il \emph{tempo tecnico} è il tempo a disposizione del giocatore nell'esecuzione
di una sequenza percezione--elaborazione--esecuzione.
\end{libro}

