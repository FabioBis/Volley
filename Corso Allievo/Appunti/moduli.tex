\chapter*{Moduli del Corso\markboth{Moduli del Corso}{}}

\begin{enumerate}
\item Metodologia dell’allenamento sportivo.
\item Lo sviluppo delle capacità fisiche nella pallavolo - Obiettivi iniziali relativi alla preparazione fisica con i giovani.
\item Principi della programmazione tecnica per le fasce d'età giovanili - Principi metodologici nell'organizzazione del settore giovanile.
\item Cenni sul rapporto tra motricità ed accrescimento - L'apprendimento motorio.
\item Le caratteristiche dei movimenti generali e specifici della pallavolo.
\item Dalla didattica del palleggio alla identificazione delle attitudini per il ruolo di alzatore.
\item Dalla didattica del bagher alla identificazione delle attitudini per il ruolo di ricevitore.
\item La didattica della schiacciata.
\item Dalla didattica della schiacciata alla identificazione delle tecniche di attacco caratteristiche dei vari ruoli.
\item La didattica del servizio e suo sviluppo.
\item Dalla didattica del bagher alla impostazione delle tecniche di difesa.
\item Dalla didattica del muro alla identificazione delle tecniche per zone di competenza.
\item L'identificazione del centrale.
\item I modelli di prestazione dell'under 14.
\item I modelli di prestazione dell'under 16.
\item Identificazione dei criteri di specificità dei modelli di prestazione - identificazione dei criteri di differenziazione tra settore maschile e settore femminile.
\item I modelli di prestazione dell'under 18.
\item L'esercizio di sintesi.
\item L'allenamento tecnico-tattico attraverso il gioco.
\item Principali traumi del pallavolista - primo soccorso, taping funzionale.
\item Regole di gioco Tecnica arbitrale e compilazione referto.
\end{enumerate}