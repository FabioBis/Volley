\chapter{Lo sviluppo delle capacità fisiche nella pallavolo}

\section{Allenamento della forza}
Si sente spesso parlare della potenza degli atleti, ma esattamente cosa si
intende? La \emph{potenza} può essere definita come \emph{forza} x
\emph{velocità}.

\begin{defi}
La \emph{forza} è la capacità motoria dell'uomo che permette di vincere una resistenza o di opporvisi con un impegno tensivo della muscolatura.
\end{defi}

Elementi dell'espressione della forza sono: fattori strutturali (la composizione
del muscolo), fattori nervosi (gli impulsi elettrici) e fattori elastici (fisici).
Il corpo deve essere mantenuto in equilibrio sia su questi fattori che sul
rapporto forza e velocità che esprime la potenza, occorre lavorare sulla forza
in base all'età.

\subsection{Fattori strutturali}
Il \emph{Muscolo scheletrico} è composto da vari \emph{fascicoli muscolari}
che a loro volta sono composti dalle \emph{fibre muscolari}.

Gli \emph{elementi contrattili} del muscolo sono le fibre muscolari, la loro
lunghezza (numero dei \emph{sacromeri}\footnote{Il sacromero è la più piccola
unità del muscolo in grado di contrarsi.}) incide sulla velocità di contrazione.

Per quanto riguarda l'aspetto quantitativo delle fibre si parla di
\emph{ipertrofia}. Durante l'allenamento le fibre più deboli subiscono più
danni, queste vengono rigenerate con fibre più forti e quindi più spesse
(aumento della forza tensiva). L'allenamento sulla muscolatura è basato su
tempi di recupero brevi.

parlando di \emph{tipologia delle fibre}, si possono classificare le fibre
in base alla
loro velocità: abbiamo le fibre rosse (o di tipo I) più lente e le fibre bianche
(o di tipo IIb) più veloci. Più aumenta il carico di lavoro e più lavorano le
fibre bianche, viceversa quelle rosse. Per la pallavolo le fibre più importanti
sono quelle bianche, che lavorano sfruttando poco ossigeno (lavoro anaerobico), consumando più zuccheri e non producono acido lattico (lavoro alattacido). 
%Il lavoro anaerobico alattacido produce Adenosin Tri-fosfato (ATP). 

\begin{ross_es}
Il metabolismo energetico più utilizzato nella pallavolo è quello
\emph{anaerobico alattacido} caratterizzato dalla mancanza richiesta di ossigeno
e dalla mancata produzione di acido lattico.
\end{ross_es}

\subsection{Fattori nervosi}
\begin{itemize}
\item[-] Reclutamento delle fibre;
\item[-] Coordinazione intramuscolare (lavoro all'interno del singolo muscolo)
  \begin{itemize}
  \item[$\cdot$]Sincronizzazione delle unità motorie;
  \item[$\cdot$]Regolazione della frequenza degli impulsi;
  \end{itemize}
\item[-] Coordinazione intermuscolare (lavoro di tutti i muscoli)
  \begin{itemize}
  \item[$\cdot$]Coordinazione dei muscoli \emph{agonisti} -- \emph{antagonisti};
  \item[$\cdot$]Coordinazione dei muscoli ausiliari.
  \end{itemize}
\end{itemize}

\section{Attivazione e forza a carico naturale \\(Bonfatti)}

Occorre partire dal materiale umano che si ha a disposizione.
Dopo una giornata tra scuola e casa in
cui mantengono per ore la stessa postura i ragazzi arrivano in palestra che sono rigidi, il ché innesca
dei problemi importanti; il ginocchio resta flesso, tutti i muscoli posteriori della coscia sono corti,
mentre la richiesta della nostra attività è basata su salti e spostamenti che prevedono degli
allungamenti. Da qua per forza di cose insorgono stati infiammatori.

L'obiettivo di una buona
attivazione è quello di sbloccare e dare un'elasticità quasi naturale ad un corpo che è rimasto in una posizione statica per tanto tempo.

Altro obiettivo dell'attivazione (e allungamento finale) è quello di compensare quelle posture
dinamiche che la pallavolo richiede, soprattutto i movimenti tipici dello sport che sono
prevalentemente di chiusura (muro, bagher, schiacciata portano alla chiusura del dorso,
soprattutto se abbiamo un pallavolista alto, i banchi di scuola sono bassi.
Si devono mobilizzare i segmenti e dare elasticità soprattutto al tronco.
Gli esercizi di compensazione sulle tenute e defaticamento finale.

Partiamo con l'attivazione a terra perchè le articolazioni sono più libere che in piedi (colonna in
scarico)a cui abbinare degli esercizi di spostamento e di motricità per lo sviluppo delle capacità
pallavolistiche (andature correttamente eseguite).
\begin{enumerate}
\item[-]Addominali e dorsali non solo fatti in scarico a terra, ma da in piedi con diverse proposte educative
per il tronco a supportare il movimento delle braccia;
\item[-]Andature classiche e non (adatte creare i presupposti per i movimenti specifici);
\item[-]Forza a carico naturale;
\item[-]Palle mediche;
\item[-]Bilancieri e manubri;
\item[-]Pesi.
\end{enumerate}

Gli angoli specifici della pallavolo sono quelli su cui dovremmo andare ad esprimere forza. La coordinazione
intrasegmentaria deve partire dalla capacità del tronco di supportare slanci e spinte degli arti superiori e
inferiori.

Con il nostro carico di allenamenti $40-45$ minuti di preparazione fisica sono sufficienti.

\subsection{Lavoro a terra ($10$')}
A terra ci si deve stare poco, ma ci sono le esigenze sopraccitate per anche, ginocchia, zona lombare che
devono lavorare in condizioni di scarico della colonna. Gli esercizi si presentano a carico naturale, la spalla,
nei lavori a terra invece è in condizioni di carico (si verificano spesso condizioni di richiesta di forza
isometrica e in allungamento).
\begin{enumerate}
\item Anche: mobilizzazioni del bacino (15-16 movimenti)
\item Allungamento adduttori (farfalla) e per la fascia lata (avvicinando i talloni al bacino, che può essere
anche un test: questa postura, se crea difficoltà, implica dei problemi di mobilita dell'anca).
Soprattutto in fase di inizio preparazione in quanto se cala il tono del quadricipite comincia a
lavorare la parte esterna.
\item Torsioni a terra supine e prone in cui l'atleta deve percepire soprattutto la rotazione del bacino
(espirazione in torsione, in quanto il diaframma si rilassa e permette la rotazione vertebrale)
\item Ginocchio distensioni coordinate ed alternate (una specie di leg extension controllata)
\item Addominali e dorsali a terra. Gli addominali e dorsali come abbiamo sempre fatto (tipo il crunch)
tolgono delle tensioni invece noi dobbiamo andare a creare una cintura di tensioni che vadano ad
insistere sul bacino e scaricano la colonna al fine di preservare dallo stress da salto e spostamento.
Fase di flessione dev'essere più lunga di quella di apertura, l'importante è che non si salga dritti, ma
che si diminuisca la distanza tra sterno e pube. 20” - 30”
\item Concetto dello spiedino: recupero attivo e per non accumulare troppo acido lattico, si alterna un
esercizio per la parete anteriore ed uno per quella posteriore, in cui la pausa non esiste perché
lavorano i muscoli dorsali mentre recuperano gli addominali.
\item Parte alta del dorso, scapolari e romboide: adduzioni per le scapole. Sono esercizi per la mobilità
articolare, se si vogliono fissare le spalle servono i pesi, però le bambine fanno fatica così.
\item Addome: tutti gli addominali lavorano in sinergia meccanica, quindi non si riescono ad escludere,
ma si cambia solamente la percentuale di muscolatura di addome che si attiva. Gli addominali
devono essere fatti in distensione e non solo dall'orizzontale, in quanto un allenamento composto
solamente da questa tipologia di addominali può portare allo strappo del retto.
\item Dorso: orologio a terra che obbliga le scapole a ruotare in modo armonico ed allo stesso tempo a
rimanere adesa al dorso; estensioni simmetriche (alternate e coordinate) eseguite in modo lento, in
modo da creare quella tonicità atta a supportare l'allenamento tecnico. No aperture per non
sovraccaricare il dorsale che fa venire il dorso curvo.
\begin{itemize}
\item[a] Esercizi di slancio in quadrupedia (propedeutici per gli appoggi e difesa)
\item[b] Esercizi dinamici di slancio in quadrupedia per il controllo del corpo
 propedeutici per la difesa (stella e calciate)
\end{itemize}
\item Alternare gli esercizi per addominali: le doppie chiusure e le torsioni. Conviene alternarle, una volta
si lavora solo sui retti e l'altra solo sulle torsioni (serie da 30 ripetizioni);
\item Non vanno proposte delle tenute da 1 minuto (vanno fatte alla fine dell'allenamento) perché il
nostro sport richiede tutt'altra dinamica. Il cervelletto si è abituato a togliere delle scosse di
attivazione per tenere fermo il muscolo, poi si parte con dei movimenti ad alto carico eccentrico e
arrivano impulsi intensi.
\item Legge dell'alternanza nel reclutamento del muscolo e lavoro ad alto peso con stimolo dato dalla
risposta ormonale a distanza;
\item Attivazione mentale;
\item Quindi si parte sulla mobilità in piedi.
\item Lavoro sulle caviglie, controllo dei piedi;
\item Ginocchia: propedeutica per lo squat;
\item Mobilizzazione dell'anca (pendolo);
\item Esercizio globale per anca e ginocchio: l'affondo, sia sagittale che frontale che in torsione (se si ha
un certo fondo). 20”. Con i molleggi eseguiti lentamente stimolano i vasti negli angoli specifici.
\item Aggiungere il lavoro di coordinazione sul tronco ed abituare l'atleta alla frenata eccentrica.
\item Lavorare sulla flesso estensione del tronco (immaginarsi in movimento di srotolamento della frusta)
in cui gli angoli si aprono in maniera coordinata;
\item Dorso: solo in aperture. Stessa postura di prima, senza schiena dritta (per non forzare la posizione
dei nuclei intravertebrali). Slanci alternati con le velocità specifiche dell'attacco con
sensibilizzazione della spalla e del gomito. I muscoli profondi (rotatori della spalla, sovraspinoso)
vengono educati a mantenere la spalla stabile e quindi a controllare il movimento. Adduzioni. Sono
esercizi coordinativi.
\item Circonduzioni, sempre in postura di spinta, in apertura.
\item Servono per educare alcuni muscoli a fissare le spalle, tipo orologio e rotazioni.
\item Collo, poco, torsioni e flessioni laterali.
\end{enumerate}

\subsubsection{Riassumendo}
\begin{itemize}
\item[-]Rullate e spinte dei piedi;
\item[-]Affondi piccoli;
\item[-]Spinta del ginocchio;
\item[-]Lavoro propriocettivo con salto frontale e laterale;
\item[-]Lavoro coordinato affondi e tronco;
\item[-]Flesso-estensione;
\item[-]Spalle.
\end{itemize}


\section{Forza a corpo libero}
Abilità nel gestire il salto e le fasi di volo sono gli obiettivi di questi esercizi.
Forza concentrica: funzione tonica.
\subsection{Polpaccio}
\begin{enumerate}
\item Spinta del polpaccio (i gemelli hanno una funzione protettiva del ginocchio): l'importante è
mantenere il corpo compatto, sia in appoggio monopodalico che bipodalico. Lo si può fare sia
lentamente che in modo dinamico, con le braccia in appoggio. Va gestita l'ESTENSIONE. (fascia
che s'innesta nel medio gluteo, isolando la catena si migliora l'estensione) per cui anche nella corsa
si deve aprire l'anca. Per completare bene le spinte bisogna far lavorare il gluteo. Si possono usare
le superfici instabili in modo da sollecitare anche i propriocettori (che sono quegli elementi che
lavorano in modo qualitativo sulle articolazioni
\item Posizione di spinta da posizione di difesa, si sollecita particolarmente il soleo: spinta + mobilità
articolare delle caviglie + sensibilità e mantenimento della postura. 3 x 10.
\end{enumerate}
\subsection{Coscia -- Anca}
\begin{enumerate}
\item Salita e discesa dalla panca: inizio anno 3 x 10 salita con una gamba sola (braccia dietro la schiena in
modo che il corpo si metta automaticamente in asse di spinta evitando le compensazioni). Si
alterna il piede di salita e quello di discesa. Si parte con un ritmo lento per poi passare ad
un'esecuzione dinamica anche balzato o con lo stacco alternato (richiamo del ginocchio come
imitativo della fast). Questo non serve per la forza esplosiva per cui serve la pressa/slitta, ma serve
per educare ed avere un reclutamento abbastanza rapido (in difesa soprattutto nelle uscite in
avanti in cui di solito si sviluppano dei contromovimenti). 3 x 8/12 per esecuzione lenta, serie da
5/6 ripetizioni per un movimento più esplosivo. Contropiegate 3 x 10;
\item Anca: per fare lo squat prima bisogna vedere che problemi ci sono. Se si hanno i flessori corti o
rigidità lombare (che di solito sono collegati) si può fare in appoggio monopodalico per mandare in
estensione l'anca.
\end{enumerate}
\subsection{Addominali}
\begin{enumerate}
\item Lavori in sospensione: appesi alla spalliera: richiamo alternato delle ginocchia –- richiamo delle
gambe sopra la testa –- oppure in flessione laterale: oscillazioni e pendolo destra e sinistra. Si può
fare lavoro educativo degli arti inferiori sul tronco (richiami alternati delle gambe e torsioni).
\end{enumerate}
\subsection{Dorsali}
\begin{enumerate}
\item Prima di tutto bisogna capire se vanno elasticizzati o meno, per cui prima una sessione di
allungamento qualora ce ne sia bisogno. Non dobbiamo pensare solo ai lunghi, ma capire che ci
debba essere una struttura funzionale (vedere se ho i dorsali forti ma i flessori deboli etc… per cui
qualche test per vedere se ci sia bisogno di lavoro specifico prima di partire con gli esercizi). Noi
non abbiamo bisogno come i lanciatori del disco o del martello di proteggere la schiena (sono
muscoli posturali e quindi sono perennemente in retrazione); le attività esplosive tendono ad
accorciare i muscoli, per cui se un atleta si trova a stare molto in piedi quando va a fare
allenamento sembra di spostare una montagna. Ecco perché a volte è meglio solo lavorare in
allungamento e non fare un lavoro dinamico;
\item Flessioni sulla cavallina;
\item Slanci in quadrupedia e lavori indiretti.
\end{enumerate}
\subsection{Arti superiori}
\begin{enumerate}
\item Tricipiti: spinta al muro con pallone (capolungo del tricipite);
\item Piegamenti sulle braccia (da fare con progressività):
\begin{enumerate}
\item Partire dall'appoggio sulle ginocchia (aspetto scapolare, tenute finché la scapola non parte
lateralmente) e tenute isometriche con differenti distanze del petto da terra.
\item Piegamenti in ceduta;
\item Discesa fino a $90^{o}$ al gomito;
\item Fino a terra;
\item Piegamenti a gambe larghe;
\item A gomiti vicino al busto (meno sollecitato il capolungo del bicipite);
\item A gambe vicine;
\end{enumerate}
\item Piegamenti sulla panca per tricipiti;
\item Il "bruco";
\item Palla sotto le ginocchia e distensioni delle gambe;
\item Piegamenti sui palloni;
\item Rotolamenti sul pallone.
\end{enumerate}

\subsubsection{
Piccolo circuito di esercizi per cui diventa immediato il transfer per affinità di movimento e di reclutamento delle fibre}
\begin{enumerate}
\item Squat su tavoletta con mani dietro la schiena o con bastoncino + salto imitativo del muro, con
rimbalzo, a parete con palla medica;
\item Tricipiti al muro 3x10 + lanci per elasticità con palla medica da sopra la testa;
\item Piegamenti sulle braccia con mani a terra o sui palloni + imitativo di tacco e slancio con palla
medica.
\end{enumerate}

\begin{ross}
In una seduta di allenamento è sconsigliabile fare stretching prima delle
attività
fisiche pesanti. Quelli di stretching sono infatti esercizi
vasocostrittori\footnote{Che restringono i vasi sanguigni, diminuendo l'apporto
di ossigeno ai muscoli.} e farli prima aumenta il rischio di infortunio.
\end{ross}

\section{Circuito ad alta sinergia muscolare}
\subsection{Manubri}
Maschi 4/5 Kg mentre le femmine 2/3 Kg. Dopo molto lavoro del deltoide subentra più stress per i muscoli profondi (rotatori).
\begin{enumerate}
\item Step-up con elevazione delle braccia (alzare il ginocchio opposto), alternare la gamba che spinge.
Cercare di dare un condizionamentoe e stabilità generale, senza avere molta similitudine con la
gestualità della tecnica specifica. Si può studiare un circuito che richiami il gesto tecnico per poi
avere un transfer maggiore. \`E utile da alternare a quello generale perché ovvia al problema della
monotonia. L'unico contro è quello di fare movimenti troppo simili della pallavolo (elementi che
presuppongono lo svincolamento del bacino).
\item Affondi diagonali tipo a “v” + imitativo delle braccia in entrata in ricezione. L'affondo
è un movimento che presuppone un adattamento nel gioco. Può essere fatto anche su superfici
instabili (automaticamente si sistemano le posture)
\item Squat + imitativo dello slancio delle braccia a muro (instabilità delle spalle)
\item Affondo laterale e alzata laterale (spalle aperte e scapole vicine). Utilizzare la “respirazione
di forza” e non fisiologica (per bloccare il diaframma) che si usa nelle palestre di fitness.
L'importante è che gli atleti non siano in apnea.
\item Tirate al mento.
\item Affondo avanti ed elevazione avanti con manubri: si lavora sempre su spinta e frenata +
stabilità articolare del ginocchio, mentre le spalle non devono ingobbirsi (come nel bagher). Si crea
anche una costruzione di una base metabolica in quanto nel nostro sport gli scambi lunghi esistono;
\item Squat giù più alzata frontale con bilanciere: le prime volte si chiede che la presa sia comoda
e simmetrica; c'è controllo della zona lombare. Una volta che si domina l'aspetto coordinativo la
grossa discriminante è la velocità esecuzione (quindi importanti accelerazioni). Rapporto tra pesisti
e velocisti.
\item Affondo in avanti con torsione del busto con bilanciere e ritorno. Non meno di 10 –- 12 e
non più di 16. (ca. 20'');
\item Squat giù e lento dietro con bilanciere: veloce la spinta e lenta la discesa.
\item Affondo laterale più bicipide con bilanciere (il bicipite è uno stabilizzatore della spalla), con
controllo della schiena;
\item Spinta dei piedi più pull-over con manubri: presa dei manubri incrociata e spinte verso l'alto
finendo sulla punta dei piedi (gomito abbastanza chiusi);
\item Squat più rematore con manubri.
\end{enumerate}

\subsection{Palla Medica}
\begin{itemize}
\item[-] Discorso preventivo;
\item[-] Accelerare un attrezzo.
\end{itemize}
Si parla prevalentemente di condizionamento dell'attacco.

\begin{defi}\emph{Atletismo}: atletizzare e rendere atleticamente abile un giocatore per il proprio sport.
\end{defi}
Obiettivo è sempre il controllo del tronco.
\begin{enumerate}
\item Caricamento e salto sul posto con slancio della palla medica sopra alla testa.
\item A coppie, presa della palla medica in affondo e uscita dalla posizione porgendo la palla al
compagno.
\item Giavellottista al muro 4 – 5 lanci per ogni piede.
\item Passo – stacco e lancio della palla medica (2 kg);
\item Presa della palla medica da terra, portarla al petto e lancio;
\item Lancio dorsale. (concatenazione della catena posteriore per la spinta in apertura)
\item Presa in salto (1 – 2 kg). Uno lavora e l'altro assiste. 5 -6 a testa e cambio.
\item Imitativo della fast con slancio della palla medica per avanti – alto.
\end{enumerate}

