\chapter{Metodologia dell'allenamento sportivo}

L'allenamento sportivo è un processo pedagogico-educativo complesso. Si concretizza con l'organizzazione dell'esercizio fisico ripetuto in quantità e intensità tali da produrre carichi crescenti, che stimolino i processi fisiologici di supercompensazione e migliorino le capacità dell'atleta (fisiche, psichiche, tattiche e tecniche).

Cosa serve per insegnare a giocare?
Occorre valutare aspetti: condizionali/coordinativi, tecnici, tattici e psicologici.

\begin{libro}
Il corso allievo--allenatore costituisce il primo step formativo e conferisce
un grado di conoscenza adeguato per le seguenti problematiche:
\emph{programmazione} (sez. \ref{programmazione}),
\emph{sviluppo tecnico} (sez. \ref{svtecnico}),
\emph{sviluppo tattico} (sez. \ref{svtattico})
e \emph{preparazione fisica} (cap. \ref{fisico}).
\end{libro}

\section{Obiettivi}
\begin{itemize}
\item Curare la formazione delle funzioni fisiche, l'espressione della motricità e della personalità.
\item Compensare la mancanza di movimento del moderno sistema di vita.
\item Facilitare l'ampliamento di abilità e comportamenti motori.
\item Promuovere l'interesse per le attività sportive.
\item Migliorare le capacità di prestazione.
\end{itemize}

La Metodologia si basa sulla conoscenza dei principi dell'allenamento, sulla Programmazione, sui Mezzi e sulle Capacità.

\section{Principi dell'allenamento}
Tutela della salute ed unità tra formazione fisica ed educazione globale.
Unità tra carico e recupero.
Gradualità, continuità, successione dei carichi.
Sistematicità dell'insegnamento.
Consapevolezza, autonomia, autostima.
Comunicazione.
Stabilità dei risultati.


\section{Programmazione}\label{programmazione}
\`E necessario partire dal gioco. Chi alleno? Cosa succede in partita? Come devo/posso giocare? E di conseguenza cosa devo migliorare.
\begin{ross}
La pianificazione e la programmazione vengono fatte per obiettivi. Si divide
l'anno sportivo in:
\begin{enumerate}
\item[-]Macrociclo: su più mesi, periodo preparatorio, periodo agonistico
(campionato) e periodo transitorio (sosta tra due campionati).
\item[-]Mesociclo: mensile.
\item[-]Microciclo: settimanale.
\item[-]Singola seduta d'allenamento.
\end{enumerate}
\end{ross}

\subsubsection{Analisi del gioco}
Modello prestativo (cosa mi serve per vincere nella categoria che faccio?).
Competitività.
Fasi della partita: ricezione--punto, battuta--punto.

\subsubsection{Definire le priorità}
Qual è la cosa che migliora il gioco? Cosa insegnare per primo ad un giocatore? Quale chiave di lettura dare ad un fondamentale nel contesto del gioco? Quali sono le caratteristiche essenziali per giocare subito e bene?

\subsubsection{Definire gli obiettivi}
Programma motorio --- sviluppo capacità organico muscolari.
Fondamentali e tecniche di riferimento.
Specializzazione dei ruoli.
Orientamento e motivazione.
Sistema di gioco.
Specializzazione dell'allenamento.

Gli obiettivi devono essere \emph{stimolanti} (curiosità e interesse condiviso),
\emph{raggiungibili} (percezione di poter raggiungere l'obiettivo),
\emph{realistici} (concreti, non astratti) e \emph{specifici} (caratteristici
della pallavolo).

\begin{libro}
L'allievo allenatore dovrà dimostrare di essere in grado di gestire una sorta
di \emph{eserciziario} e una serie di \emph{protocolli esemplificativi} da
riproporre nel lavoro in palestra.
\end{libro}

\section{Sviluppo Tecnico}\label{svtecnico}
Sia la formazione tecnica individuale che quella di squadra sono il risultato della qualità del lavoro svolto. La corretta applicazione della didattica, lo studio e la ripetizione (corretta) dei gesti tecnici nella fase dell'apprendimento motorio (unitamente allo sviluppo della motricità specifica) sono di fondamentale importanza nell'età 8--13 anni.
Gli interventi sulla correzione degli errori propriamente definiti saranno oggetto di lavoro successivo, nei 14--16 anni.
Occorre fare attenzione sulle differenze tra maschile e femminile (relativamente allo sviluppo).
Gli elementi della formazione tecnica sono: didattica e capacità di imitazione del gesto tecnico (fase di apprendimento), volume e qualità del lavoro,  consapevolezza--motivazione--mentalità, correzione degli errori e rinforzo, capacità di gestione e figura dell'allenatore, il gioco (modulo, sviluppo del concetto tattico).


\subsection{Strumenti}
Lavoro analitico e super--analitico.
Lavoro sintetico (alto contenuto tattico e tecnico).
Lavoro globale (con indicazioni tattiche e tecniche).
Gioco.

Gli esercizi possono essere utilizzati per imparare, correggere, migliorare e fissare. Hanno tutti un comune denominatore: molte ripetizione corrette nel tempo.

\subsubsection{Fasi di apprendimento motorio}
In queste fasi è importante seguire i concetti base della didattica: dal facile al difficile, dal semplice al complesso, dal poco al molto e dal conosciuto allo sconosciuto.

\subsubsection{Strumenti}
Esercizi di sensibilizzazione e di controllo.
Esecuzione del gesto globale per imitazione.
Analisi tecnica (scomposizione del gesto tecnico) e lavoro analitico.
Propriocettività (capacità di percepire e riconoscere la posizione del proprio corpo).
Sviluppo motricità specifica.
Lavoro in condizione facilitata.
Ripetizioni --- rinforzo.
Multilateralità.
Precisione.
Destrezza.
Giochi.


\subsection{Concetti generali per lo sviluppo della tecnica}
Pianificazione dell'allenamento: tema principale, secondario e permanente. L'obiettivo deve essere ben definito, l'atleta deve sapere sempre cosa non può sbagliare, quasi mai sbagliare e quali sono gli errori trascurabili. L'obiettivo deve essere raggiungibile e diverso in relazione alla capacità degli atleti.
Sviluppo dell'equilibrio  tra i fondamentali delle due fasi di gioco: cambio palla (ricezione) e break--point (servizio--contrattacco).
Occorre dare un ritmo adeguato all'allenamento, l'intensità non deve essere a scapito della qualità. Sviluppare la motricità specifica e delle posture, con e senza palla.
Gli esercizi a rotazione con diversi fondamentali hanno scarsa efficacia sul fissare e rinforzare i fondamentali specifici. Non fermare mai la palla, tranne che nella correzione dell'errore e nella didattica di partenza (lavoro analitico).
Proporre nuovi temi o esercizi impegnativi sotto l'aspetto dell'attenzione all'inizio dell'allenamento (sopratutto nell'analitico).
Analizzare il fondamentale in forma globale sopratutto quando questo è già strutturato in forma corretta nelle sue parti, altrimenti si corre il rischio di allenare “adattamenti” personali
ed abitudini sbagliate.

Esercitazione sui fondamentali.
A posizioni fisse/rotazione a tempo.
A serie e ripetizioni.
Ad obiettivo raggiunto.
A tempo con obiettivo raggiunto.
Di gioco con obiettivi tecnici mirati (ad obiettivo raggiunto).
Giochi a punteggio speciale.

\subsection{Correzione degli errori}
Occorre utilizzare un modello tecnico di riferimento, in base al quale definire ed osservare errori primari ed errori secondari. Bisogna far prendere coscienza all'atleta dell'errore di esecuzione e motivarlo (deve essere il suo problema). Si utilizza il lavoro analitico per lo sviluppo della capacità motoria e di rinforzo ed il lavoro globale in situazione facilitata.
\`E utile utilizzare sussidi particolari, supporti video e parole chiave che richiamino il concetto.

\subsection{Contestualizzare l'errore}
Incoraggiare e fornire rinforzi positivi sul miglioramento. Correggere un errore per volta per focalizzare l'attenzione dell'atleta. Deve essere un lavoro analitico quotidiano.
Le esercitazioni analitiche devono essere quasi individuali o per piccoli gruppi.
\`E indispensabile la consapevolezza dell'errore e che il desiderio di migliorare sia reale.
Abituare i ragazzi ad essere esigenti con se stessi, non accettare l'errore e sopratutto non giustificarlo. Occorre disciplina e partecipazione.

\section{Sviluppo Tattico}\label{svtattico}
Elaborazione tattica individuale ed anticipazione motoria.
Indurre al ragionamento (imparare a giocare).
Mentalità agonistica (il gusto del gioco).
Guidare l'alzatore con punti di riferimento precisi per la distribuzione.
Progressività nel modulo di gioco.
Il “ruolo” non come obiettivo da raggiungere ma come strumento di crescita tecnica.
La squadra giovanile deve cercare di eliminare i difetti, non di nasconderli.
Il miglioramento tecnico permette e favorisce il miglioramento tattico e non viceversa, il modulo di gioco deve essere adeguato alla capacità tecniche e deve rispecchiare gli obiettivi posti in allenamento.
Insegnare a giocare il 6 vs 6 applicando sempre il concetto di “prima-dopo-durante”.
Sviluppo delle situazioni di gioco in base alla loro correlazione: senza punteggio ma studiando e fermando il gioco quando serve, oppure con il raggiungimento di obiettivi prefissati.
Giochi a punteggio speciale.

\subsection{Sviluppo del programma motorio}
Specificità dei transfer: capacità di saper trovare una risposta motoria in un contesto nuovo, utilizzando quanto appreso in precedenza, in altre azioni o esercizi.
Le progressioni devono essere in numero limitato ed il più possibili simili alle situazioni di gioco.

\subsection{Feedback sull'esercizio}
La posizione dei giocatori in campo, i loro movimenti ed il loro orientamento rispetto alla rete.
La sequenza degli eventi e le coordinazione dei tempi nella sequenza, la reazione agli stimoli in un giocatore, lo svolgimento logico di un'azione di gioco.

\subsection{Organizzazione dell'esercizio}
I giocatori non coinvolti devono assicurarsi che la continuità ed il ritmo dell'esercizio vengano mantenuti, i palloni che rotolano o che rimbalzano non mettano in pericolo o disturbino gli altri giocatori impegnati attivamente nell'esercizio. I carrelli dei palloni siano sempre pieni e si trovino nel posto giusto, l'allenatore sia rifornito adeguatamente in modo che si possa concentrare sui fondamentali che sta eseguendo e sul controllo dell'esercizio.
I giocatori devono apprendere e controllare le capacità tecniche necessarie per svolgere le funzioni di supporto: raccolta e fornitura dei palloni e fare da riferimento.

\subsection{Metodo globale}
Vantaggi: più motivante, transfer diretto nei confronti del gioco, si impara a giocare, feedback più reale e possibile. 
Svantaggi: quantità delle ripetizioni, riuscire a seguire tutti i giocatori, stabilire delle priorità, usare concetti chiave, poter correggere.

I principi essenziali:
la regola prima di tutto,
chiarezza degli obiettivi,
coerenza negli interventi,
pretesa ed alta richiesta portano alla qualificazione.

\section{Riscaldamento}
Il riscaldamento è una pratica eseguita prima della prestazione fisica--sportiva per consentire al corpo di riuscire ad affrontare il vero e proprio allenamento nelle migliori condizioni possibili, preparandolo, migliorando la prestazione fisica e riducendo il rischio di infortuni.

Più il lavoro centrale dell'allenamento  sarà intenso (carico) più lungo
dovrà risultare il riscaldamento.

\section{Cardio Frequenzimetro e formula di\\ Karvonen}
La formula di Karvonen è un procedimento empirico utilizzato nel campo della medicina dello sport per misurare il parametro di intensità nell'esercizio cardiovascolare e per pianificare l'allenamento sportivo sulla base della frequenza cardiaca.

Per poter individuare la percentuale dell'intensità massima da applicare durante l'allenamento occorre misurare la frequenza cardiaca basale (F.C.B) del giocatore
(ovvero la frequenza cardiaca a riposo) ed applicare le seguenti formule,
se per esempio si vuole lavorare al $65$\%:
\begin{flushleft}
 \[
\textit{Maschile:} \quad
 \left( 220 - \textit{età} - \textit{F.C.B} \right) \cdot 0,65 + \textit{F.C.B}
 \]
\end{flushleft}
\begin{flushleft}
 \[
\textit{Femminile:} \quad
 \left( 226 - \textit{età} - \textit{F.C.B} \right) \cdot 0,65 + \textit{F.C.B}
 \]
\end{flushleft}