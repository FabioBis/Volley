\hyphenation{
Bonfatti
}


\documentclass[a4paper, 12pt]{book}
\usepackage{fancyhdr}
\pagestyle{fancy}
% i comandi seguenti impediscono la scrittura in maiuscolo
% dei nomi dei capitoli e dei paragrafi nelle intestazioni
\renewcommand{\chaptermark}[1]{\markboth{#1}{}}
\renewcommand{\sectionmark}[1]{\markright{\thesection\ #1}}
\fancyhf{} % rimuove l’attuale contenuto dell’intestazione
% e del pi\‘e di pagina
\fancyhead[LE,RO]{\bfseries\thepage}
\fancyhead[LO]{\bfseries\rightmark}
\fancyhead[RE]{\bfseries\leftmark}
\renewcommand{\headrulewidth}{0.5pt}
\renewcommand{\footrulewidth}{0pt}
\addtolength{\headheight}{0.5pt} % riserva spazio per la linea
\fancypagestyle{plain}{%
\fancyhead{} % ignora, nello stile plain, le intestazioni
\renewcommand{\headrulewidth}{0pt} % e la linea
}
\usepackage[italian]{babel}
\usepackage[utf8]{inputenc}
\usepackage{amsthm}
\usepackage{vhistory}
\usepackage{hyperref}

\title{Appunti Corso Allievo Allenatore 2015/2016}
\author{
%Biselli Fabio
%\and
%Burgazzi Elisa 
%\and
%Cappellini Chiara
%\and
%Do Carmo Luciana
%\and
%Giovanelli Enrica
%\and
%Longhi Corrado
%\and
%Mazzari Francesco
%\and
%Monfasani Andrea
%\and
%Nedeljkovic Jovana
%\and
%Schenardi Simone
%\and
%Tommaso Cardinali
%\and
%Valentino Martina
}
\date {}

\frenchspacing



\newtheorem*{ross}{Appunti (Rossetti)}
\newtheorem*{ross_es}{Possibile argomento d'esame (Rossetti)}
\newtheorem*{libro}{Manuale FIPAV}

\theoremstyle{remark}
\newtheorem*{nota}{Nota}

\theoremstyle{definition}
\newtheorem{defi}{Definizione}

\begin{document}
{\samepage
\maketitle
\begin{versionhistory}
  \vhEntry{0.1}{11/05/2016}{Allievi 2015/2016}{Prima bozza}
  \vhEntry{0.2}{22/05/2016}{Allievi 2015/2016}{Bozza Modulo 1}
  \vhEntry{0.3}{24/05/2016}{Allievi 2015/2016}{Moduli del corso, Bozza Modulo 5}
  \vhEntry{0.4}{25/05/2016}{Allievi 2015/2016}{Bozza Scouts}
  \vhEntry{0.5}{27/05/2016}{Allievi 2015/2016}{Bozza Modulo 2}
  \vhEntry{0.6}{29/05/2016}{Allievi 2015/2016}{Forza}
\end{versionhistory}
}

\tableofcontents

\chapter*{Moduli del Corso\markboth{Moduli del Corso}{}}

\begin{enumerate}
\item Metodologia dell'allenamento sportivo.
\item Lo sviluppo delle capacità fisiche nella pallavolo - Obiettivi iniziali relativi alla preparazione fisica con i giovani.
\item Principi della programmazione tecnica per le fasce d'età giovanili - Principi metodologici nell'organizzazione del settore giovanile.
\item Cenni sul rapporto tra motricità ed accrescimento - L'apprendimento motorio.
\item Le caratteristiche dei movimenti generali e specifici della pallavolo.
\item Dalla didattica del palleggio alla identificazione delle attitudini per il ruolo di alzatore.
\item Dalla didattica del bagher alla identificazione delle attitudini per il ruolo di ricevitore.
\item La didattica della schiacciata.
\item Dalla didattica della schiacciata alla identificazione delle tecniche di attacco caratteristiche dei vari ruoli.
\item La didattica del servizio e suo sviluppo.
\item Dalla didattica del bagher alla impostazione delle tecniche di difesa.
\item Dalla didattica del muro alla identificazione delle tecniche per zone di competenza.
\item L'identificazione del centrale.
\item I modelli di prestazione dell'under 14.
\item I modelli di prestazione dell'under 16.
\item Identificazione dei criteri di specificità dei modelli di prestazione - identificazione dei criteri di differenziazione tra settore maschile e settore femminile.
\item I modelli di prestazione dell'under 18.
\item L'esercizio di sintesi.
\item L'allenamento tecnico-tattico attraverso il gioco.
\item Principali traumi del pallavolista - primo soccorso, taping funzionale.
\item Regole di gioco Tecnica arbitrale e compilazione referto.
\end{enumerate}
\chapter{Metodologia dell'allenamento sportivo}

L'allenamento sportivo è un processo pedagogico-educativo complesso. Si concretizza con l'organizzazione dell'esercizio fisico ripetuto in quantità e intensità tali da produrre carichi crescenti, che stimolino i processi fisiologici di supercompensazione e migliorino le capacità dell'atleta (fisiche, psichiche, tattiche e tecniche).

Cosa serve per insegnare a giocare?
Occorre valutare aspetti: condizionali/coordinativi, tecnici, tattici e psicologici.

\begin{libro}
Il corso allievo--allenatore costituisce il primo step formativo e conferisce
un grado di conoscenza adeguato per le seguenti problematiche:
\emph{programmazione} (sez. \ref{programmazione}),
\emph{sviluppo tecnico} (sez. \ref{svtecnico}),
\emph{sviluppo tattico} (sez. \ref{svtattico})
e \emph{preparazione fisica} (cap. \ref{fisico}).
\end{libro}

\section{Obiettivi}
\begin{itemize}
\item Curare la formazione delle funzioni fisiche, l'espressione della motricità e della personalità.
\item Compensare la mancanza di movimento del moderno sistema di vita.
\item Facilitare l'ampliamento di abilità e comportamenti motori.
\item Promuovere l'interesse per le attività sportive.
\item Migliorare le capacità di prestazione.
\end{itemize}

La Metodologia si basa sulla conoscenza dei principi dell'allenamento, sulla Programmazione, sui Mezzi e sulle Capacità.

\section{Principi dell'allenamento}
Tutela della salute ed unità tra formazione fisica ed educazione globale.
Unità tra carico e recupero.
Gradualità, continuità, successione dei carichi.
Sistematicità dell'insegnamento.
Consapevolezza, autonomia, autostima.
Comunicazione.
Stabilità dei risultati.


\section{Programmazione}\label{programmazione}
\`E necessario partire dal gioco. Chi alleno? Cosa succede in partita? Come devo/posso giocare? E di conseguenza cosa devo migliorare.
\begin{ross}
La pianificazione e la programmazione vengono fatte per obiettivi. Si divide
l'anno sportivo in:
\begin{enumerate}
\item[-]Macrociclo: su più mesi, periodo preparatorio, periodo agonistico
(campionato) e periodo transitorio (sosta tra due campionati).
\item[-]Mesociclo: mensile.
\item[-]Microciclo: settimanale.
\item[-]Singola seduta d'allenamento.
\end{enumerate}
\end{ross}

\subsubsection{Analisi del gioco}
Modello prestativo (cosa mi serve per vincere nella categoria che faccio?).
Competitività.
Fasi della partita: ricezione--punto, battuta--punto.

\subsubsection{Definire le priorità}
Qual è la cosa che migliora il gioco? Cosa insegnare per primo ad un giocatore? Quale chiave di lettura dare ad un fondamentale nel contesto del gioco? Quali sono le caratteristiche essenziali per giocare subito e bene?

\subsubsection{Definire gli obiettivi}
Programma motorio --- sviluppo capacità organico muscolari.
Fondamentali e tecniche di riferimento.
Specializzazione dei ruoli.
Orientamento e motivazione.
Sistema di gioco.
Specializzazione dell'allenamento.

Gli obiettivi devono essere \emph{stimolanti} (curiosità e interesse condiviso),
\emph{raggiungibili} (percezione di poter raggiungere l'obiettivo),
\emph{realistici} (concreti, non astratti) e \emph{specifici} (caratteristici
della pallavolo).

\begin{libro}
L'allievo allenatore dovrà dimostrare di essere in grado di gestire una sorta
di \emph{eserciziario} e una serie di \emph{protocolli esemplificativi} da
riproporre nel lavoro in palestra.
\end{libro}

\section{Sviluppo Tecnico}\label{svtecnico}
Sia la formazione tecnica individuale che quella di squadra sono il risultato della qualità del lavoro svolto. La corretta applicazione della didattica, lo studio e la ripetizione (corretta) dei gesti tecnici nella fase dell'apprendimento motorio (unitamente allo sviluppo della motricità specifica) sono di fondamentale importanza nell'età 8--13 anni.
Gli interventi sulla correzione degli errori propriamente definiti saranno oggetto di lavoro successivo, nei 14--16 anni.
Occorre fare attenzione sulle differenze tra maschile e femminile (relativamente allo sviluppo).
Gli elementi della formazione tecnica sono: didattica e capacità di imitazione del gesto tecnico (fase di apprendimento), volume e qualità del lavoro,  consapevolezza--motivazione--mentalità, correzione degli errori e rinforzo, capacità di gestione e figura dell'allenatore, il gioco (modulo, sviluppo del concetto tattico).


\subsection{Strumenti}
Lavoro analitico e super--analitico.
Lavoro sintetico (alto contenuto tattico e tecnico).
Lavoro globale (con indicazioni tattiche e tecniche).
Gioco.

Gli esercizi possono essere utilizzati per imparare, correggere, migliorare e fissare. Hanno tutti un comune denominatore: molte ripetizione corrette nel tempo.

\subsubsection{Fasi di apprendimento motorio}
In queste fasi è importante seguire i concetti base della didattica: dal facile al difficile, dal semplice al complesso, dal poco al molto e dal conosciuto allo sconosciuto.

\subsubsection{Strumenti}
Esercizi di sensibilizzazione e di controllo.
Esecuzione del gesto globale per imitazione.
Analisi tecnica (scomposizione del gesto tecnico) e lavoro analitico.
Propriocettività (capacità di percepire e riconoscere la posizione del proprio corpo).
Sviluppo motricità specifica.
Lavoro in condizione facilitata.
Ripetizioni --- rinforzo.
Multilateralità.
Precisione.
Destrezza.
Giochi.


\subsection{Concetti generali per lo sviluppo della tecnica}
Pianificazione dell'allenamento: tema principale, secondario e permanente. L'obiettivo deve essere ben definito, l'atleta deve sapere sempre cosa non può sbagliare, quasi mai sbagliare e quali sono gli errori trascurabili. L'obiettivo deve essere raggiungibile e diverso in relazione alla capacità degli atleti.
Sviluppo dell'equilibrio  tra i fondamentali delle due fasi di gioco: cambio palla (ricezione) e break--point (servizio--contrattacco).
Occorre dare un ritmo adeguato all'allenamento, l'intensità non deve essere a scapito della qualità. Sviluppare la motricità specifica e delle posture, con e senza palla.
Gli esercizi a rotazione con diversi fondamentali hanno scarsa efficacia sul fissare e rinforzare i fondamentali specifici. Non fermare mai la palla, tranne che nella correzione dell'errore e nella didattica di partenza (lavoro analitico).
Proporre nuovi temi o esercizi impegnativi sotto l'aspetto dell'attenzione all'inizio dell'allenamento (sopratutto nell'analitico).
Analizzare il fondamentale in forma globale sopratutto quando questo è già strutturato in forma corretta nelle sue parti, altrimenti si corre il rischio di allenare “adattamenti” personali
ed abitudini sbagliate.

Esercitazione sui fondamentali.
A posizioni fisse/rotazione a tempo.
A serie e ripetizioni.
Ad obiettivo raggiunto.
A tempo con obiettivo raggiunto.
Di gioco con obiettivi tecnici mirati (ad obiettivo raggiunto).
Giochi a punteggio speciale.

\subsection{Correzione degli errori}
Occorre utilizzare un modello tecnico di riferimento, in base al quale definire ed osservare errori primari ed errori secondari. Bisogna far prendere coscienza all'atleta dell'errore di esecuzione e motivarlo (deve essere il suo problema). Si utilizza il lavoro analitico per lo sviluppo della capacità motoria e di rinforzo ed il lavoro globale in situazione facilitata.
\`E utile utilizzare sussidi particolari, supporti video e parole chiave che richiamino il concetto.

\subsection{Contestualizzare l'errore}
Incoraggiare e fornire rinforzi positivi sul miglioramento. Correggere un errore per volta per focalizzare l'attenzione dell'atleta. Deve essere un lavoro analitico quotidiano.
Le esercitazioni analitiche devono essere quasi individuali o per piccoli gruppi.
\`E indispensabile la consapevolezza dell'errore e che il desiderio di migliorare sia reale.
Abituare i ragazzi ad essere esigenti con se stessi, non accettare l'errore e sopratutto non giustificarlo. Occorre disciplina e partecipazione.

\section{Sviluppo Tattico}\label{svtattico}
Elaborazione tattica individuale ed anticipazione motoria.
Indurre al ragionamento (imparare a giocare).
Mentalità agonistica (il gusto del gioco).
Guidare l'alzatore con punti di riferimento precisi per la distribuzione.
Progressività nel modulo di gioco.
Il “ruolo” non come obiettivo da raggiungere ma come strumento di crescita tecnica.
La squadra giovanile deve cercare di eliminare i difetti, non di nasconderli.
Il miglioramento tecnico permette e favorisce il miglioramento tattico e non viceversa, il modulo di gioco deve essere adeguato alla capacità tecniche e deve rispecchiare gli obiettivi posti in allenamento.
Insegnare a giocare il 6 vs 6 applicando sempre il concetto di “prima-dopo-durante”.
Sviluppo delle situazioni di gioco in base alla loro correlazione: senza punteggio ma studiando e fermando il gioco quando serve, oppure con il raggiungimento di obiettivi prefissati.
Giochi a punteggio speciale.

\subsection{Sviluppo del programma motorio}
Specificità dei transfer: capacità di saper trovare una risposta motoria in un contesto nuovo, utilizzando quanto appreso in precedenza, in altre azioni o esercizi.
Le progressioni devono essere in numero limitato ed il più possibili simili alle situazioni di gioco.

\subsection{Feedback sull'esercizio}
La posizione dei giocatori in campo, i loro movimenti ed il loro orientamento rispetto alla rete.
La sequenza degli eventi e le coordinazione dei tempi nella sequenza, la reazione agli stimoli in un giocatore, lo svolgimento logico di un'azione di gioco.

\subsection{Organizzazione dell'esercizio}
I giocatori non coinvolti devono assicurarsi che la continuità ed il ritmo dell'esercizio vengano mantenuti, i palloni che rotolano o che rimbalzano non mettano in pericolo o disturbino gli altri giocatori impegnati attivamente nell'esercizio. I carrelli dei palloni siano sempre pieni e si trovino nel posto giusto, l'allenatore sia rifornito adeguatamente in modo che si possa concentrare sui fondamentali che sta eseguendo e sul controllo dell'esercizio.
I giocatori devono apprendere e controllare le capacità tecniche necessarie per svolgere le funzioni di supporto: raccolta e fornitura dei palloni e fare da riferimento.

\subsection{Metodo globale}
Vantaggi: più motivante, transfer diretto nei confronti del gioco, si impara a giocare, feedback più reale e possibile. 
Svantaggi: quantità delle ripetizioni, riuscire a seguire tutti i giocatori, stabilire delle priorità, usare concetti chiave, poter correggere.

I principi essenziali:
la regola prima di tutto,
chiarezza degli obiettivi,
coerenza negli interventi,
pretesa ed alta richiesta portano alla qualificazione.

\section{Riscaldamento}
Il riscaldamento è una pratica eseguita prima della prestazione fisica--sportiva per consentire al corpo di riuscire ad affrontare il vero e proprio allenamento nelle migliori condizioni possibili, preparandolo, migliorando la prestazione fisica e riducendo il rischio di infortuni.

Più il lavoro centrale dell'allenamento  sarà intenso (carico) più lungo
dovrà risultare il riscaldamento.

\section{Cardio Frequenzimetro e formula di\\ Karvonen}
La formula di Karvonen è un procedimento empirico utilizzato nel campo della medicina dello sport per misurare il parametro di intensità nell'esercizio cardiovascolare e per pianificare l'allenamento sportivo sulla base della frequenza cardiaca.

Per poter individuare la percentuale dell'intensità massima da applicare durante l'allenamento occorre misurare la frequenza cardiaca basale (F.C.B) del giocatore
(ovvero la frequenza cardiaca a riposo) ed applicare le seguenti formule,
se per esempio si vuole lavorare al $65$\%:
\begin{flushleft}
 \[
\textit{Maschile:} \quad
 \left( 220 - \textit{età} - \textit{F.C.B} \right) \cdot 0,65 + \textit{F.C.B}
 \]
\end{flushleft}
\begin{flushleft}
 \[
\textit{Femminile:} \quad
 \left( 226 - \textit{età} - \textit{F.C.B} \right) \cdot 0,65 + \textit{F.C.B}
 \]
\end{flushleft}
\chapter{Lo sviluppo delle capacità fisiche nella pallavolo}

% \\ Obiettivi iniziali relativi alla preparazione fisica con i giovani}
\chapter{Le caratteristiche dei movimenti generali e specifici}

La Pallavolo è uno sport di situazione caratterizzato da movimenti aciclici (che
non si ripetono in modo sistematico nel tempo), non stereotipati (l'ambiente
esterno influisce su di essi) e tattici. \`E uno sport “open skills”, ovvero uno
sport in cui le abilità sono messe in evidenza da un ambiente variabile e poco
predicibile (influenzati da eventi esterni: gli avversari). Nella Pallavolo non
c'è possesso palla.

\begin{libro}
Per disciplina \emph{open skills} si intende una disciplina caratterizzata da
\emph{abilità motorie aperte}, ossia costantemente utilizzate in risposta
alla variabilità situazionale.
\end{libro}

\subsubsection{I Fondamentali}
I vari comportamenti sequenziali che i giocatori attuano durante le azioni
di gioco rappresentano i \emph{fondamentali}, essi sono:
\begin{itemize}
\item[-]Ricezione (azione di cambiopalla);
\item[-]Alzata (azione di cambiopalla);
\item[-]Attacco e copertura (azione di cambiopalla);
\item[-]Servizio (azione di contrattacco);
\item[-]Muro (azione di contrattacco);
\item[-]Difesa (azione di contrattacco);
\item[-]Ricostruzione e attacco con copertura (azione di contrattacco).
\end{itemize}

\subsubsection{Tecniche di Base}
\begin{itemize}
\item[-]Palleggio;
\item[-]Bagher;
\item[-]Battuta float piedi a terra;
\item[-]Schiacciata;
\item[-]Alcune tecniche di spostamento.
\end{itemize}



\section{Metodologia}
Si può classificare l'esercizio (il mezzo) in vario modo:
\begin{itemize}
\item[-]Analitico, sintetico, globale;
\item[-]Ciclico: il medesimo gesto tecnico ripetuto;
\item[-]Guidato: si comunica al giocatore la variazione;
\item[-]Situazionale: si simulano situazioni di gioco (poca prevedibilità).
\end{itemize}


\section{Programmazione}
La programmazione dipende principalmente dal dalla categoria che si deve
allenare. Nel minivolley fino all'under 13 ci si basa sulla motricità e sul
gioco della pallavolo, mettendo le basi. Nell'under 14--18 si lavora
principalmente sulla tecnica.

\section{Prestazione sportiva}
Si basa sulle capacità coordinative, condizionali e di elaborazione tattica.
Queste vengono sviluppate attraverso la metodologia e la programmazione
dell'allenamento sportivo mediante il principio di adattamento.

\subsection{Capacità Coordinative}
Le capacità coordinative, deputate principalmente all'apprendimento, al 
controllo e alla regolazione del movimento, sono le seguenti:
\begin{itemize}
\item[-]\emph{Apprendimento} del movimento;
\item[-]\emph{Controllo} del movimento;
\item[-]\emph{Adattamento} e \emph{trasformazione} del movimento;
\item[-]Tutte le \emph{manifestazioni coordinative secondarie}.
\end{itemize}

\subsection{Capacità Condizionali}
Sono caratteristiche del movimento della pallavolo, fanno riferimento a 
tutti i meccanismi energetici, strutturali, fisiologici e funzionali del
movimento (capacità organico--muscolari), queste sono:
\begin{itemize}
\item[-]\emph{Forza}: definita nel capitolo \ref{fisico};
\item[-]\emph{Rapidità}: la capacità di reagire ad uno stimolo con la massima
velocità possibile;
\item[-]\emph{Resistenza}:  la capacità di mantenere una
 determinata prestazione, esiste quella generale (aerobica, marginale nella
 pallavolo) e quella speciale (resistenza specifica di gara).
\end{itemize}


\section{Complessità dei sistemi}
Controllo delle posture, i processi mentali ed anticipatori analizzati gli
stimoli. Intercettazione (tocco della palla) e cambio direzione e dell'altezza
(variazione di traiettoria della palla). Lateralità (emisfero destro/sinistro).

\subsection{Dominanza}
Un emisfero comanda le dominanze e non sempre c'è un uso simmetrico del corpo.
La \emph{lateralizzazione} comincia ad essere netta dopo i $5$-$6$ anni e
non procede con la stessa velocità su tutti gli organi e le strutture. Ad
esempio i piedi e gli occhi subiscono un ritardo rispetto alla mano.

Risulta importante (nel minivolley) identificare le dominanze di ogni individuo
ed assecondarle, cercare di correggerle può dare disturbi alla lettura, alla
motricità, all'orientamento, etc.

Nella pallavolo ci sono movimenti/fondamentali simmetrici e non, in quelli
simmetrici si hanno meno problemi.

\subsection{Accelerazione e Decelerazione}
Nella pallavolo il concetto di velocità deve essere correlato a quello di
decelerazione. Il sistema accelerazione/decelerazione (enfatizzato
nell'intercettazione della palla) coinvolge altri sistemi: posture statiche e
dinamiche, processi di elaborazione e anticipazione degli stimoli ambientali,
lateralità e modulazione e frequenza del ritmo dei movimenti.

\subsection{Traiettorie}
La conoscenza delle traiettorie avviene per \emph{prove ed errori} e la
misura del processo di apprendimento è la \emph{stabilità} dell'efficacia
motoria, ovvero non ci sono oscillazioni nel numero di esecuzioni corrette e
nel tipo di errore negli esercizi.
Risulta fondamentale imparare a leggere le traiettorie già dal minivolley.

Si evidenziano due fattori condizionanti per la lettura delle traiettorie
ed i movimenti per l'intercettazione: i \emph{posizionamenti} ed il
\emph{concetto di tempo tecnico}.

\begin{libro}
Il \emph{tempo tecnico} è il tempo a disposizione del giocatore nell'esecuzione
di una sequenza percezione--elaborazione--esecuzione.
\end{libro}


\chapter{Introduzione allo Scouting}

Lo scouting è l'azione di rilevamento dei i dati generanti da una partita o da un allenamento.
I dati servono da supporto alla creazione delle statistiche e quindi alle analisi successive, queste confermano o smentiscono le “sensazioni dell’allenatore”, fondate sulla sua
esperienza ed abilità.
L’atleta informato risulta essere più motivato, lo scout aiuta ad eliminare la cultura dell’alibi.

La rilevazione e l'analisi dei dati devono rispondere ai principi dell'oggettività e dell'obiettività.
Il giocatore deve essere valutato per il suo reale rendimento senza condizionamenti emotivi (simpatia o antipatia), senza condizionamenti legati alla spettacolarità di certe azioni.
Lo souting deve essere basato su criteri di valutazione attentamente studiati e codificati dall'allenatore, a cui lo scout-man deve fare fede.

\subsubsection{A cosa serve lo scout?}
Conoscere le caratteristiche dalla proprio squadra;
Conoscere le caratteristiche delle squadre avversarie;
Ottenere dati precisi e misurabili (oggettivi);
Comunicare ed analizzare i dati alla squadra.

\section{Come si effettua lo scouting?}
L'allenatore definisce a priori, per ogni fondamentale, i criteri di valutazione in base alle proprie esigenze, ogni fondamentale può avere un numero variabile di valutazioni.
Lo scouting può essere fatto in tempo reale (durante la partita o allenamento), in qual caso si rende necessaria la figura di scout-man, oppure mediante la registrazione della partita.
Nel primo caso, l'allenatore può usufruire dei dati in tempo reale ed eventualmente intervenire adeguatamente, nel secondo i dati saranno disponibili solo per una seconda analisi.
Di seguito viene definito un sistema basato su 5 criteri di valutazione (simboli o gradi di performance), che risulta tra i più utilizzati. Nulla vieta di variarne i criteri di valutazione o di limitare i fondamentali.

I simboli qui sotto riportati sono i seguenti: $\sharp$ (doppio positivo), $+$ (positivo),
$/$ (slash è un simbolo speciale), $-$ (negativo), $--$ (doppio negativo). A volte viene utilizzato
un altro simbolo speciale ($!$).

\subsubsection{Battuta}
\begin{enumerate}
\item[$\sharp$] Ace, 0, 1 o 2 tocchi dell'avversario;
\item[$+$] Battuta positiva, il pallone ritorna direttamente nel campo della squadra al servizio o la squadra che riceve non può attaccare;
\item[$/$] Battuta “slash”, la squadra che riceve ha un attacco obbligato: palla oltre i 4 metri (gioco obbligato sulle bande) o palla troppo attaccata a rete (gioco obbligato in primo tempo);
\item[$-$] Battuta negativa, la squadra che riceve ha tutte le opzioni d'attacco;
\item[$--$] Errore.
\end{enumerate}

\subsubsection{Ricezione}
\begin{enumerate}
\item[$\sharp$] Ricezione perfetta, il palleggiatore ha la palla entro un raggio di al massimo 1,5 mt ad un altezza adeguata per avere tutte le opzioni d'attacco;
\item[$+$] Ricezione positiva, il palleggiatore ha la palla ad un raggio di poco più di 1/1,5 mt (massimo 2,5) ad un altezza adeguata senza avere il gioco al centro;
\item[$/$] Ricezione “slash”, consente un solo tipo di alzata, oltre i 4 metri o troppo attaccata a rete;
\item[$-$] Ricezione negativa, non consente l'attacco (free-ball);
\item[$--$] Errore, ace subito (0, 1 o 2 tocchi);
\end{enumerate}

\subsubsection{Attacco}
\begin{enumerate}
\item[$\sharp$] Attacco punto;
\item[$+$] Attacco positivo, la squadra che difende non può contrattaccare (compreso il muro rigiocato dalla squadra che ha attaccato);
\item[$/$] Attacco “slash”, la squadra che difende riesce a ricostruire;
\item[$-$] Attacco negativo, la squadra che difende ha tutte le opzioni d'attacco o ha ottenuto il punto a muro;
\item[$--$] Errore diretto;
\end{enumerate}

Spesso si registrano anche le direzioni degli attacchi mediante appositi grafici.

\subsubsection{Muro}
\begin{enumerate}
\item[$\sharp$] Muro punto;
\item[$+$] Muro passivo, favorisce la squadra che difende che può contrattacare agevolmente, oppure torna nel campo della squadra che ha attaccato ma che non può ricostruire un attacco (free-ball);
\item[$/$] Muro ininfluente, la squadra che ha attaccato può rigiocare agevolmente;
\item[$-$] Mani-out;
\item[$--$] Invasione a rete;
\end{enumerate}

\subsubsection{Difesa}
\begin{enumerate}
\item[$\sharp$] Difesa difficile, acrobatica, lontana o con un gesto tecnico evidente;
\item[$+$] Difesa positiva, consente la ricostruzione di un'azione;
\item[$/$] Copertura dell'attacco;
\item[$-$] Difesa negativa, non consente la ricostruzione;
\item[$--$] Difesa non tenuta (tocco errato) o errore di posizione;
\end{enumerate}


\subsection{Formule}
In questa sezione vengono definite le sole tre formule date a lezione e che possono essere richieste durante l'esame, queste riguardano la positività in attacco ed in ricezione e l'efficienza in attacco. Si definiscono per ogni giocatore le seguenti quantità: $A_{\sharp}$ il numero degli attacchi punto, $A_{-}$ il numero degli attacchi negativi, $A_{--}$ il numero degli errori in attacco, $A_{T}$ il numero totale degli attacchi, $R_{\sharp}$ il numero di ricezioni perfette, $R_{+}$ il numero delle ricezioni positive ed $R_{T}$ il numero totale delle ricezioni.

\begin{flushleft}
 \[
\textbf{Efficienza in Attacco:} \quad
 \frac{(A_{\sharp} - A_{-} - A_{--})}{A_{T}} \cdot 100
 \]
\end{flushleft}
 
\begin{flushleft}
\[
\textbf{Positività in Attacco:} \quad
 \frac{A_{\sharp}}{A_{T}} \cdot 100
\]
\end{flushleft}

\begin{center}
\[
\textbf{Positività in Ricezione:} \quad
 \frac{(R_{\sharp} + R_{+})}{R_{T}} \cdot 100
\]
\end{center}


\end{document}