\hyphenation{
Bonfatti
}


\documentclass[a4paper, 12pt]{book}
\usepackage{fancyhdr}
\pagestyle{fancy}
% i comandi seguenti impediscono la scrittura in maiuscolo
% dei nomi dei capitoli e dei paragrafi nelle intestazioni
\renewcommand{\chaptermark}[1]{\markboth{#1}{}}
\renewcommand{\sectionmark}[1]{\markright{\thesection\ #1}}
\fancyhf{} % rimuove l’attuale contenuto dell’intestazione
% e del pi\‘e di pagina
\fancyhead[LE,RO]{\bfseries\thepage}
\fancyhead[LO]{\bfseries\rightmark}
\fancyhead[RE]{\bfseries\leftmark}
\renewcommand{\headrulewidth}{0.5pt}
\renewcommand{\footrulewidth}{0pt}
\addtolength{\headheight}{0.5pt} % riserva spazio per la linea
\fancypagestyle{plain}{%
\fancyhead{} % ignora, nello stile plain, le intestazioni
\renewcommand{\headrulewidth}{0pt} % e la linea
}
\usepackage[italian]{babel}
\usepackage[utf8]{inputenc}
\usepackage{amsthm}
\usepackage{vhistory}
\usepackage{hyperref}

\title{Appunti Corso Allievo Allenatore 2015/2016}
\author{
%Biselli Fabio
%\and
%Burgazzi Elisa 
%\and
%Cappellini Chiara
%\and
%Do Carmo Luciana
%\and
%Giovanelli Enrica
%\and
%Longhi Corrado
%\and
%Mazzari Francesco
%\and
%Monfasani Andrea
%\and
%Nedeljkovic Jovana
%\and
%Schenardi Simone
%\and
%Tommaso Cardinali
%\and
%Valentino Martina
}
\date {}

\frenchspacing



\newtheorem*{ross}{Appunti (Rossetti)}
\newtheorem*{ross_es}{Possibile argomento d'esame (Rossetti)}

\theoremstyle{remark}
\newtheorem*{nota}{Nota}

\theoremstyle{definition}
\newtheorem{defi}{Definizione}

\begin{document}
{\samepage
\maketitle
\begin{versionhistory}
  \vhEntry{0.1}{11/05/2016}{Allievi 2015/2016}{Prima bozza}
  \vhEntry{0.2}{22/05/2016}{Allievi 2015/2016}{Bozza Modulo 1}
  \vhEntry{0.3}{24/05/2016}{Allievi 2015/2016}{Moduli del corso, Bozza Modulo 5}
  \vhEntry{0.4}{25/05/2016}{Allievi 2015/2016}{Bozza Scouts}
  \vhEntry{0.5}{27/05/2016}{Allievi 2015/2016}{Bozza Modulo 2}
  \vhEntry{0.6}{29/05/2016}{Allievi 2015/2016}{Forza}
\end{versionhistory}
}

\tableofcontents

\hyphenation{
Bonfatti
Karvonen
Tarasconi
Rossetti
Maffi
}

\chapter*{Moduli del Corso\markboth{Moduli del Corso}{}}

\begin{itemize}
\item[Mod. 01:]La seduta di allenamento Tecnico-tattico. \emph{[Rossetti]}
\item[Mod. 02:]L'esercizio analitico, sintetico e globale nel sistema di allenamento tecnico-tattico. \emph{[Rossetti]}
\item[Mod. 03:]La forza nella pallavolo. \emph{[Rossetti]}
\item[Mod. 04:]La gestione del gruppo. \emph{[Maffi]}
\item[Mod. 05:]La valutazione. \emph{[Maffi]}
\item[Mod. 06:]Percorso didattico per l'impostazione delle tecniche di alzata. \emph{[Maffi]}
\item[Mod. 07:]Percorso didattico per l'impostazione delle tecniche di ricezione. \emph{[Maffi]}
\item[Mod. 08:]L'attacco secondo i modelli esecutivi caratteristici dei vari ruoli. \emph{[Maffi]}
\item[Mod. 09:]Percorso didattico per l'impostazione delle tecniche di muro. \emph{[Maffi]}
\item[Mod. 10:]Percorso didattico per l'impostazione delle tecniche degli interventi difensivi. \emph{[Maffi]}
\item[Mod. 11:]Contenuti dell'allenamento specifico dell'alzatore. \emph{[Tarasconi]}
\item[Mod. 12:]Contenuti dell'allenamento specifico del ricevitore attaccante. \emph{[Tarasconi]}
\item[Mod. 13:]Contenuti dell'allenamento specifico del centrale. \emph{[Tarasconi]}
\item[Mod. 14:]Contenuti dell'allenamento specifico dell'opposto. \emph{[Tarasconi]}
\item[Mod. 15:]Contenuti dell'allenamento specifico del libero. \emph{[Tarasconi]}
\item[Mod. 16:]L'esercizio di battuta-ricezione e l'allenamento del sistema di ricezione. \emph{[Tarasconi]}
\item[Mod. 17:]L'allenamento del sistema tattico per l'azione di cambio-palla. \emph{[Tarasconi]}
\item[Mod. 18:]L'allenamento del sistema di muro e del collegamento battuta-muro. \emph{[Tarasconi]}
\item[Mod. 19:]Lo sviluppo della tecnica attraverso l'esercizio di difesa e ricostruzione. \emph{[Rossetti]}
\item[Mod. 20:]Lo sviluppo della tecnica attraverso l'esercizio di attacco e contro muro. \emph{[Rossetti]}
\item[Mod. 21:]L'allenamento dei sistemi di difesa e contrattacco. \emph{[Rossetti]}
\item[Mod. 22:]L'allenamento dei sistemi di copertura e contrattacco. \emph{[Rossetti]}
\item[Mod. 23:]La differenziazione metodologica tra esercizi di sintesi  per l'allenamento del cambio palla e della ricostruzione. \emph{[Maffi]}
\item[Mod. 24:]L'allenamento tattico attraverso il gioco. \emph{[Maffi]}
\item[Mod. 25:]Protocolli per l'allenamento della forza. \emph{[Maffi]}
\item[Mod. 26:]Fisiologia riferita al carico di lavoro e agli adattamenti conseguenti. \emph{[Rossetti]}
\item[Mod. 27:]Prevenzione delle possibili patologie da sovraccarico. \emph{[Rossetti]}
\item[Mod. 28:]Tecniche di rilevamento dei dati durante le partite. \emph{[Rossetti]}
\item[Mod. 29:]Analisi e utilizzo della scoutizzazione effettuata in palestra. \emph{[Rossetti]}
\item[Mod. 30:]Approfondimenti sul Regolamento e sulla Tecnica Arbitrale. \emph{[Arbitri]}
\item[Mod. 31:]Il Sitting Volley. \emph{[Raho]}
\end{itemize}
\chapter{Metodologia dell'allenamento sportivo}

L'allenamento sportivo è un processo pedagogico-educativo complesso. Si concretizza con l'organizzazione dell'esercizio fisico ripetuto in quantità e intensità tali da produrre carichi crescenti, che stimolino i processi fisiologici di supercompensazione e migliorino le capacità dell'atleta (fisiche, psichiche, tattiche e tecniche).

Cosa serve per insegnare a giocare?
Occorre valutare aspetti: condizionali/coordinativi, tecnici, tattici e psicologici.

\begin{libro}
Il corso allievo--allenatore costituisce il primo step formativo e conferisce
un grado di conoscenza adeguato per le seguenti problematiche:
\emph{programmazione} (sez. \ref{programmazione}),
\emph{sviluppo tecnico} (sez. \ref{svtecnico}),
\emph{sviluppo tattico} (sez. \ref{svtattico})
e \emph{preparazione fisica} (cap. \ref{fisico}).
\end{libro}

\section{Obiettivi}
\begin{itemize}
\item Curare la formazione delle funzioni fisiche, l'espressione della motricità e della personalità.
\item Compensare la mancanza di movimento del moderno sistema di vita.
\item Facilitare l'ampliamento di abilità e comportamenti motori.
\item Promuovere l'interesse per le attività sportive.
\item Migliorare le capacità di prestazione.
\end{itemize}

La Metodologia si basa sulla conoscenza dei principi dell'allenamento, sulla Programmazione, sui Mezzi e sulle Capacità.

\section{Principi dell'allenamento}
Tutela della salute ed unità tra formazione fisica ed educazione globale.
Unità tra carico e recupero.
Gradualità, continuità, successione dei carichi.
Sistematicità dell'insegnamento.
Consapevolezza, autonomia, autostima.
Comunicazione.
Stabilità dei risultati.


\section{Programmazione}\label{programmazione}
\`E necessario partire dal gioco. Chi alleno? Cosa succede in partita? Come devo/posso giocare? E di conseguenza cosa devo migliorare.
\begin{ross}
La pianificazione e la programmazione vengono fatte per obiettivi. Si divide
l'anno sportivo in:
\begin{enumerate}
\item[-]Macrociclo: su più mesi, periodo preparatorio, periodo agonistico
(campionato) e periodo transitorio (sosta tra due campionati).
\item[-]Mesociclo: mensile.
\item[-]Microciclo: settimanale.
\item[-]Singola seduta d'allenamento.
\end{enumerate}
\end{ross}

\subsubsection{Analisi del gioco}
Modello prestativo (cosa mi serve per vincere nella categoria che faccio?).
Competitività.
Fasi della partita: ricezione--punto, battuta--punto.

\subsubsection{Definire le priorità}
Qual è la cosa che migliora il gioco? Cosa insegnare per primo ad un giocatore? Quale chiave di lettura dare ad un fondamentale nel contesto del gioco? Quali sono le caratteristiche essenziali per giocare subito e bene?

\subsubsection{Definire gli obiettivi}
Programma motorio --- sviluppo capacità organico muscolari.
Fondamentali e tecniche di riferimento.
Specializzazione dei ruoli.
Orientamento e motivazione.
Sistema di gioco.
Specializzazione dell'allenamento.

Gli obiettivi devono essere \emph{stimolanti} (curiosità e interesse condiviso),
\emph{raggiungibili} (percezione di poter raggiungere l'obiettivo),
\emph{realistici} (concreti, non astratti) e \emph{specifici} (caratteristici
della pallavolo).

\begin{libro}
L'allievo allenatore dovrà dimostrare di essere in grado di gestire una sorta
di \emph{eserciziario} e una serie di \emph{protocolli esemplificativi} da
riproporre nel lavoro in palestra.
\end{libro}

\section{Sviluppo Tecnico}\label{svtecnico}
Sia la formazione tecnica individuale che quella di squadra sono il risultato della qualità del lavoro svolto. La corretta applicazione della didattica, lo studio e la ripetizione (corretta) dei gesti tecnici nella fase dell'apprendimento motorio (unitamente allo sviluppo della motricità specifica) sono di fondamentale importanza nell'età 8--13 anni.
Gli interventi sulla correzione degli errori propriamente definiti saranno oggetto di lavoro successivo, nei 14--16 anni.
Occorre fare attenzione sulle differenze tra maschile e femminile (relativamente allo sviluppo).
Gli elementi della formazione tecnica sono: didattica e capacità di imitazione del gesto tecnico (fase di apprendimento), volume e qualità del lavoro,  consapevolezza--motivazione--mentalità, correzione degli errori e rinforzo, capacità di gestione e figura dell'allenatore, il gioco (modulo, sviluppo del concetto tattico).


\subsection{Strumenti}
Lavoro analitico e super--analitico.
Lavoro sintetico (alto contenuto tattico e tecnico).
Lavoro globale (con indicazioni tattiche e tecniche).
Gioco.

Gli esercizi possono essere utilizzati per imparare, correggere, migliorare e fissare. Hanno tutti un comune denominatore: molte ripetizione corrette nel tempo.

\subsubsection{Fasi di apprendimento motorio}
In queste fasi è importante seguire i concetti base della didattica: dal facile al difficile, dal semplice al complesso, dal poco al molto e dal conosciuto allo sconosciuto.

\subsubsection{Strumenti}
Esercizi di sensibilizzazione e di controllo.
Esecuzione del gesto globale per imitazione.
Analisi tecnica (scomposizione del gesto tecnico) e lavoro analitico.
Propriocettività (capacità di percepire e riconoscere la posizione del proprio corpo).
Sviluppo motricità specifica.
Lavoro in condizione facilitata.
Ripetizioni --- rinforzo.
Multilateralità.
Precisione.
Destrezza.
Giochi.


\subsection{Concetti generali per lo sviluppo della tecnica}
Pianificazione dell'allenamento: tema principale, secondario e permanente. L'obiettivo deve essere ben definito, l'atleta deve sapere sempre cosa non può sbagliare, quasi mai sbagliare e quali sono gli errori trascurabili. L'obiettivo deve essere raggiungibile e diverso in relazione alla capacità degli atleti.
Sviluppo dell'equilibrio  tra i fondamentali delle due fasi di gioco: cambio palla (ricezione) e break--point (servizio--contrattacco).
Occorre dare un ritmo adeguato all'allenamento, l'intensità non deve essere a scapito della qualità. Sviluppare la motricità specifica e delle posture, con e senza palla.
Gli esercizi a rotazione con diversi fondamentali hanno scarsa efficacia sul fissare e rinforzare i fondamentali specifici. Non fermare mai la palla, tranne che nella correzione dell'errore e nella didattica di partenza (lavoro analitico).
Proporre nuovi temi o esercizi impegnativi sotto l'aspetto dell'attenzione all'inizio dell'allenamento (sopratutto nell'analitico).
Analizzare il fondamentale in forma globale sopratutto quando questo è già strutturato in forma corretta nelle sue parti, altrimenti si corre il rischio di allenare “adattamenti” personali
ed abitudini sbagliate.

Esercitazione sui fondamentali.
A posizioni fisse/rotazione a tempo.
A serie e ripetizioni.
Ad obiettivo raggiunto.
A tempo con obiettivo raggiunto.
Di gioco con obiettivi tecnici mirati (ad obiettivo raggiunto).
Giochi a punteggio speciale.

\subsection{Correzione degli errori}
Occorre utilizzare un modello tecnico di riferimento, in base al quale definire ed osservare errori primari ed errori secondari. Bisogna far prendere coscienza all'atleta dell'errore di esecuzione e motivarlo (deve essere il suo problema). Si utilizza il lavoro analitico per lo sviluppo della capacità motoria e di rinforzo ed il lavoro globale in situazione facilitata.
\`E utile utilizzare sussidi particolari, supporti video e parole chiave che richiamino il concetto.

\subsection{Contestualizzare l'errore}
Incoraggiare e fornire rinforzi positivi sul miglioramento. Correggere un errore per volta per focalizzare l'attenzione dell'atleta. Deve essere un lavoro analitico quotidiano.
Le esercitazioni analitiche devono essere quasi individuali o per piccoli gruppi.
\`E indispensabile la consapevolezza dell'errore e che il desiderio di migliorare sia reale.
Abituare i ragazzi ad essere esigenti con se stessi, non accettare l'errore e sopratutto non giustificarlo. Occorre disciplina e partecipazione.

\section{Sviluppo Tattico}\label{svtattico}
Elaborazione tattica individuale ed anticipazione motoria.
Indurre al ragionamento (imparare a giocare).
Mentalità agonistica (il gusto del gioco).
Guidare l'alzatore con punti di riferimento precisi per la distribuzione.
Progressività nel modulo di gioco.
Il “ruolo” non come obiettivo da raggiungere ma come strumento di crescita tecnica.
La squadra giovanile deve cercare di eliminare i difetti, non di nasconderli.
Il miglioramento tecnico permette e favorisce il miglioramento tattico e non viceversa, il modulo di gioco deve essere adeguato alla capacità tecniche e deve rispecchiare gli obiettivi posti in allenamento.
Insegnare a giocare il 6 vs 6 applicando sempre il concetto di “prima-dopo-durante”.
Sviluppo delle situazioni di gioco in base alla loro correlazione: senza punteggio ma studiando e fermando il gioco quando serve, oppure con il raggiungimento di obiettivi prefissati.
Giochi a punteggio speciale.

\subsection{Sviluppo del programma motorio}
Specificità dei transfer: capacità di saper trovare una risposta motoria in un contesto nuovo, utilizzando quanto appreso in precedenza, in altre azioni o esercizi.
Le progressioni devono essere in numero limitato ed il più possibili simili alle situazioni di gioco.

\subsection{Feedback sull'esercizio}
La posizione dei giocatori in campo, i loro movimenti ed il loro orientamento rispetto alla rete.
La sequenza degli eventi e le coordinazione dei tempi nella sequenza, la reazione agli stimoli in un giocatore, lo svolgimento logico di un'azione di gioco.

\subsection{Organizzazione dell'esercizio}
I giocatori non coinvolti devono assicurarsi che la continuità ed il ritmo dell'esercizio vengano mantenuti, i palloni che rotolano o che rimbalzano non mettano in pericolo o disturbino gli altri giocatori impegnati attivamente nell'esercizio. I carrelli dei palloni siano sempre pieni e si trovino nel posto giusto, l'allenatore sia rifornito adeguatamente in modo che si possa concentrare sui fondamentali che sta eseguendo e sul controllo dell'esercizio.
I giocatori devono apprendere e controllare le capacità tecniche necessarie per svolgere le funzioni di supporto: raccolta e fornitura dei palloni e fare da riferimento.

\subsection{Metodo globale}
Vantaggi: più motivante, transfer diretto nei confronti del gioco, si impara a giocare, feedback più reale e possibile. 
Svantaggi: quantità delle ripetizioni, riuscire a seguire tutti i giocatori, stabilire delle priorità, usare concetti chiave, poter correggere.

I principi essenziali:
la regola prima di tutto,
chiarezza degli obiettivi,
coerenza negli interventi,
pretesa ed alta richiesta portano alla qualificazione.

\section{Riscaldamento}
Il riscaldamento è una pratica eseguita prima della prestazione fisica--sportiva per consentire al corpo di riuscire ad affrontare il vero e proprio allenamento nelle migliori condizioni possibili, preparandolo, migliorando la prestazione fisica e riducendo il rischio di infortuni.

Più il lavoro centrale dell'allenamento  sarà intenso (carico) più lungo
dovrà risultare il riscaldamento.

\subsection{Cardio Frequenzimetro e formula di Karvonen}
La formula di Karvonen è un procedimento empirico utilizzato nel campo della medicina dello sport per misurare il parametro di intensità nell'esercizio cardiovascolare e per pianificare l'allenamento sportivo sulla base della frequenza cardiaca.

Per poter individuare la percentuale dell'intensità massima da applicare durante l'allenamento occorre misurare la frequenza cardiaca basale (F.C.B) del giocatore
(ovvero la frequenza cardiaca a riposo) ed applicare le seguenti formule,
se per esempio si vuole lavorare al $65$\%:
\begin{flushleft}
 \[
\textit{Maschile:} \quad
 \left( 220 - \textit{età} - \textit{F.C.B} \right) \cdot 0,65 + \textit{F.C.B}
 \]
\end{flushleft}
\begin{flushleft}
 \[
\textit{Femminile:} \quad
 \left( 226 - \textit{età} - \textit{F.C.B} \right) \cdot 0,65 + \textit{F.C.B}
 \]
\end{flushleft}
\chapter{Lo sviluppo delle capacità fisiche nella pallavolo}

\section{Allenamento della forza}
Si sente spesso parlare della potenza degli atleti, ma esattamente cosa si
intende? La \emph{potenza} può essere definita come \emph{forza} x
\emph{velocità}.

\begin{defi}
La \emph{forza} è la capacità motoria dell'uomo che permette di vincere una resistenza o di opporvisi con un impegno tensivo della muscolatura.
\end{defi}

Elementi dell'espressione della forza sono: fattori strutturali (la composizione
del muscolo), fattori nervosi (gli impulsi elettrici) e fattori elastici (fisici).
Il corpo deve essere mantenuto in equilibrio sia su questi fattori che sul
rapporto forza e velocità che esprime la potenza, occorre lavorare sulla forza
in base all'età.

\subsection{Fattori strutturali}
Il \emph{Muscolo scheletrico} è composto da vari \emph{fascicoli muscolari}
che a loro volta sono composti dalle \emph{fibre muscolari}.

Gli \emph{elementi contrattili} del muscolo sono le fibre muscolari, la loro
lunghezza (numero dei \emph{sacromeri}\footnote{Il sacromero è la più piccola
unità del muscolo in grado di contrarsi.}) incide sulla velocità di contrazione.

Per quanto riguarda l'aspetto quantitativo delle fibre si parla di
\emph{ipertrofia}. Durante l'allenamento le fibre più deboli subiscono più
danni, queste vengono rigenerate con fibre più forti e quindi più spesse
(aumento della forza tensiva). L'allenamento sulla muscolatura è basato su
tempi di recupero brevi.

La \emph{tipologia delle fibre}. Si possono classificare le fibre in base alla
loro velocità: abbiamo le fibre rosse (o di tipo I) più lente e le fibre bianche
(o di tipo IIb) più veloci. Più aumenta il carico di lavoro e più lavorano le
fibre bianche, viceversa quelle rosse. Per la pallavolo le fibre più importanti
sono quelle bianche, che lavorano sfruttando poco ossigeno (lavoro anaerobico), consumando più zuccheri e non producono acido lattico (lavoro alattacido). 
%Il lavoro anaerobico alattacido produce Adenosin Tri-fosfato (ATP). 

\begin{ross_es}
Il metabolismo energetico più utilizzato nella pallavolo è quello
\emph{anaerobico alattacido} caratterizzato dalla mancanza richiesta di ossigeno
e dalla mancata produzione di acido lattico.
\end{ross_es}

\subsection{Fattori nervosi}
\begin{itemize}
\item[-] Reclutamento delle fibre;
\item[-] Coordinazione intramuscolare (lavoro all'interno del singolo muscolo)
  \begin{itemize}
  \item[$\cdot$]Sincronizzazione delle unità motorie;
  \item[$\cdot$]Regolazione della frequenza degli impulsi;
  \end{itemize}
\item[-] Coordinazione intermuscolare (lavoro di tutti i muscoli)
  \begin{itemize}
  \item[$\cdot$]Coordinazione dei muscoli \emph{agonisti} -- \emph{antagonisti};
  \item[$\cdot$]Coordinazione dei muscoli ausiliari.
  \end{itemize}
\end{itemize}

\section{Attivazione e forza a carico naturale \\(Bonfatti)}

Occorre partire dal materiale umano che si ha a disposizione.
Dopo una giornata tra scuola e casa in
cui mantengono per ore la stessa postura i ragazzi arrivano in palestra che sono rigidi, il ché innesca
dei problemi importanti; il ginocchio resta flesso, tutti i muscoli posteriori della coscia sono corti,
mentre la richiesta della nostra attività è basata su salti e spostamenti che prevedono degli
allungamenti. Da qua per forza di cose insorgono stati infiammatori.

L'obiettivo di una buona
attivazione è quello di sbloccare e dare un'elasticità quasi naturale ad un corpo che è rimasto in una posizione statica per tanto tempo.

Altro obiettivo dell'attivazione (e allungamento finale) è quello di compensare quelle posture
dinamiche che la pallavolo richiede, soprattutto i movimenti tipici dello sport che sono
prevalentemente di chiusura (muro, bagher, schiacciata portano alla chiusura del dorso,
soprattutto se abbiamo un pallavolista alto, i banchi di scuola sono bassi.
Si devono mobilizzare i segmenti e dare elasticità soprattutto al tronco.
Gli esercizi di compensazione sulle tenute e defaticamento finale.

Partiamo con l'attivazione a terra perchè le articolazioni sono più libere che in piedi (colonna in
scarico)a cui abbinare degli esercizi di spostamento e di motricità per lo sviluppo delle capacità
pallavolistiche (andature correttamente eseguite).
\begin{enumerate}
\item[-]Addominali e dorsali non solo fatti in scarico a terra, ma da in piedi con diverse proposte educative
per il tronco a supportare il movimento delle braccia;
\item[-]Andature classiche e non (adatte creare i presupposti per i movimenti specifici);
\item[-]Forza a carico naturale;
\item[-]Palle mediche;
\item[-]Bilancieri e manubri;
\item[-]Pesi.
\end{enumerate}

Gli angoli specifici della pallavolo sono quelli su cui dovremmo andare ad esprimere forza. La coordinazione
intrasegmentaria deve partire dalla capacità del tronco di supportare slanci e spinte degli arti superiori e
inferiori.

Con il nostro carico di allenamenti $40-45$ minuti di preparazione fisica sono sufficienti.

\subsection{Lavoro a terra ($10$')}
A terra ci si deve stare poco, ma ci sono le esigenze sopraccitate per anche, ginocchia, zona lombare che
devono lavorare in condizioni di scarico della colonna. Gli esercizi si presentano a carico naturale, la spalla,
nei lavori a terra invece è in condizioni di carico (si verificano spesso condizioni di richiesta di forza
isometrica e in allungamento).
\begin{enumerate}
\item Anche: mobilizzazioni del bacino (15-16 movimenti)
\item Allungamento adduttori (farfalla) e per la fascia lata (avvicinando i talloni al bacino, che può essere
anche un test: questa postura, se crea difficoltà, implica dei problemi di mobilita dell'anca).
Soprattutto in fase di inizio preparazione in quanto se cala il tono del quadricipite comincia a
lavorare la parte esterna.
\item Torsioni a terra supine e prone in cui l'atleta deve percepire soprattutto la rotazione del bacino
(espirazione in torsione, in quanto il diaframma si rilassa e permette la rotazione vertebrale)
\item Ginocchio distensioni coordinate ed alternate (una specie di leg extension controllata)
\item Addominali e dorsali a terra. Gli addominali e dorsali come abbiamo sempre fatto (tipo il crunch)
tolgono delle tensioni invece noi dobbiamo andare a creare una cintura di tensioni che vadano ad
insistere sul bacino e scaricano la colonna al fine di preservare dallo stress da salto e spostamento.
Fase di flessione dev'essere più lunga di quella di apertura, l'importante è che non si salga dritti, ma
che si diminuisca la distanza tra sterno e pube. 20” - 30”
\item Concetto dello spiedino: recupero attivo e per non accumulare troppo acido lattico, si alterna un
esercizio per la parete anteriore ed uno per quella posteriore, in cui la pausa non esiste perché
lavorano i muscoli dorsali mentre recuperano gli addominali.
\item Parte alta del dorso, scapolari e romboide: adduzioni per le scapole. Sono esercizi per la mobilità
articolare, se si vogliono fissare le spalle servono i pesi, però le bambine fanno fatica così.
\item Addome: tutti gli addominali lavorano in sinergia meccanica, quindi non si riescono ad escludere,
ma si cambia solamente la percentuale di muscolatura di addome che si attiva. Gli addominali
devono essere fatti in distensione e non solo dall'orizzontale, in quanto un allenamento composto
solamente da questa tipologia di addominali può portare allo strappo del retto.
\item Dorso: orologio a terra che obbliga le scapole a ruotare in modo armonico ed allo stesso tempo a
rimanere adesa al dorso; estensioni simmetriche (alternate e coordinate) eseguite in modo lento, in
modo da creare quella tonicità atta a supportare l'allenamento tecnico. No aperture per non
sovraccaricare il dorsale che fa venire il dorso curvo.
\begin{itemize}
\item[a] Esercizi di slancio in quadrupedia (propedeutici per gli appoggi e difesa)
\item[b] Esercizi dinamici di slancio in quadrupedia per il controllo del corpo
 propedeutici per la difesa (stella e calciate)
\end{itemize}
\item Alternare gli esercizi per addominali: le doppie chiusure e le torsioni. Conviene alternarle, una volta
si lavora solo sui retti e l'altra solo sulle torsioni (serie da 30 ripetizioni);
\item Non vanno proposte delle tenute da 1 minuto (vanno fatte alla fine dell'allenamento) perché il
nostro sport richiede tutt'altra dinamica. Il cervelletto si è abituato a togliere delle scosse di
attivazione per tenere fermo il muscolo, poi si parte con dei movimenti ad alto carico eccentrico e
arrivano impulsi intensi.
\item Legge dell'alternanza nel reclutamento del muscolo e lavoro ad alto peso con stimolo dato dalla
risposta ormonale a distanza;
\item Attivazione mentale;
\item Quindi si parte sulla mobilità in piedi.
\item Lavoro sulle caviglie, controllo dei piedi;
\item Ginocchia: propedeutica per lo squat;
\item Mobilizzazione dell'anca (pendolo);
\item Esercizio globale per anca e ginocchio: l'affondo, sia sagittale che frontale che in torsione (se si ha
un certo fondo). 20”. Con i molleggi eseguiti lentamente stimolano i vasti negli angoli specifici.
\item Aggiungere il lavoro di coordinazione sul tronco ed abituare l'atleta alla frenata eccentrica.
\item Lavorare sulla flesso estensione del tronco (immaginarsi in movimento di srotolamento della frusta)
in cui gli angoli si aprono in maniera coordinata;
\item Dorso: solo in aperture. Stessa postura di prima, senza schiena dritta (per non forzare la posizione
dei nuclei intravertebrali). Slanci alternati con le velocità specifiche dell'attacco con
sensibilizzazione della spalla e del gomito. I muscoli profondi (rotatori della spalla, sovraspinoso)
vengono educati a mantenere la spalla stabile e quindi a controllare il movimento. Adduzioni. Sono
esercizi coordinativi.
\item Circonduzioni, sempre in postura di spinta, in apertura.
\item Servono per educare alcuni muscoli a fissare le spalle, tipo orologio e rotazioni.
\item Collo, poco, torsioni e flessioni laterali.
\end{enumerate}

\subsubsection{Riassumendo}
\begin{itemize}
\item[-]Rullate e spinte dei piedi;
\item[-]Affondi piccoli;
\item[-]Spinta del ginocchio;
\item[-]Lavoro propriocettivo con salto frontale e laterale;
\item[-]Lavoro coordinato affondi e tronco;
\item[-]Flesso-estensione;
\item[-]Spalle.
\end{itemize}


\section{Forza a corpo libero}
Abilità nel gestire il salto e le fasi di volo sono gli obiettivi di questi esercizi.
Forza concentrica: funzione tonica.
\subsection{Polpaccio}
\begin{enumerate}
\item Spinta del polpaccio (i gemelli hanno una funzione protettiva del ginocchio): l'importante è
mantenere il corpo compatto, sia in appoggio monopodalico che bipodalico. Lo si può fare sia
lentamente che in modo dinamico, con le braccia in appoggio. Va gestita l'ESTENSIONE. (fascia
che s'innesta nel medio gluteo, isolando la catena si migliora l'estensione) per cui anche nella corsa
si deve aprire l'anca. Per completare bene le spinte bisogna far lavorare il gluteo. Si possono usare
le superfici instabili in modo da sollecitare anche i propriocettori (che sono quegli elementi che
lavorano in modo qualitativo sulle articolazioni
\item Posizione di spinta da posizione di difesa, si sollecita particolarmente il soleo: spinta + mobilità
articolare delle caviglie + sensibilità e mantenimento della postura. 3 x 10.
\end{enumerate}
\subsection{Coscia -- Anca}
\begin{enumerate}
\item Salita e discesa dalla panca: inizio anno 3 x 10 salita con una gamba sola (braccia dietro la schiena in
modo che il corpo si metta automaticamente in asse di spinta evitando le compensazioni). Si
alterna il piede di salita e quello di discesa. Si parte con un ritmo lento per poi passare ad
un'esecuzione dinamica anche balzato o con lo stacco alternato (richiamo del ginocchio come
imitativo della fast). Questo non serve per la forza esplosiva per cui serve la pressa/slitta, ma serve
per educare ed avere un reclutamento abbastanza rapido (in difesa soprattutto nelle uscite in
avanti in cui di solito si sviluppano dei contromovimenti). 3 x 8/12 per esecuzione lenta, serie da
5/6 ripetizioni per un movimento più esplosivo. Contropiegate 3 x 10;
\item Anca: per fare lo squat prima bisogna vedere che problemi ci sono. Se si hanno i flessori corti o
rigidità lombare (che di solito sono collegati) si può fare in appoggio monopodalico per mandare in
estensione l'anca.
\end{enumerate}
\subsection{Addominali}
\begin{enumerate}
\item Lavori in sospensione: appesi alla spalliera: richiamo alternato delle ginocchia –- richiamo delle
gambe sopra la testa –- oppure in flessione laterale: oscillazioni e pendolo destra e sinistra. Si può
fare lavoro educativo degli arti inferiori sul tronco (richiami alternati delle gambe e torsioni).
\end{enumerate}
\subsection{Dorsali}
\begin{enumerate}
\item Prima di tutto bisogna capire se vanno elasticizzati o meno, per cui prima una sessione di
allungamento qualora ce ne sia bisogno. Non dobbiamo pensare solo ai lunghi, ma capire che ci
debba essere una struttura funzionale (vedere se ho i dorsali forti ma i flessori deboli etc… per cui
qualche test per vedere se ci sia bisogno di lavoro specifico prima di partire con gli esercizi). Noi
non abbiamo bisogno come i lanciatori del disco o del martello di proteggere la schiena (sono
muscoli posturali e quindi sono perennemente in retrazione); le attività esplosive tendono ad
accorciare i muscoli, per cui se un atleta si trova a stare molto in piedi quando va a fare
allenamento sembra di spostare una montagna. Ecco perché a volte è meglio solo lavorare in
allungamento e non fare un lavoro dinamico;
\item Flessioni sulla cavallina;
\item Slanci in quadrupedia e lavori indiretti.
\end{enumerate}
\subsection{Arti superiori}
\begin{enumerate}
\item Tricipiti: spinta al muro con pallone (capolungo del tricipite);
\item Piegamenti sulle braccia (da fare con progressività):
\begin{enumerate}
\item Partire dall'appoggio sulle ginocchia (aspetto scapolare, tenute finché la scapola non parte
lateralmente) e tenute isometriche con differenti distanze del petto da terra.
\item Piegamenti in ceduta;
\item Discesa fino a $90^{o}$ al gomito;
\item Fino a terra;
\item Piegamenti a gambe larghe;
\item A gomiti vicino al busto (meno sollecitato il capolungo del bicipite);
\item A gambe vicine;
\end{enumerate}
\item Piegamenti sulla panca per tricipiti;
\item Il "bruco";
\item Palla sotto le ginocchia e distensioni delle gambe;
\item Piegamenti sui palloni;
\item Rotolamenti sul pallone.
\end{enumerate}

\subsubsection{
Piccolo circuito di esercizi per cui diventa immediato il transfer per affinità di movimento e di reclutamento delle fibre}
\begin{enumerate}
\item Squat su tavoletta con mani dietro la schiena o con bastoncino + salto imitativo del muro, con
rimbalzo, a parete con palla medica;
\item Tricipiti al muro 3x10 + lanci per elasticità con palla medica da sopra la testa;
\item Piegamenti sulle braccia con mani a terra o sui palloni + imitativo di tacco e slancio con palla
medica.
\end{enumerate}

\begin{ross}
In una seduta di allenamento è scorretto fare stretching prima delle attività
fisiche pesanti. Quelli di stretching sono infatti esercizi
vasocostrittori\footnote{Che restringono i vasi sanguigni, diminuendo l'apporto
di ossigeno ai muscoli.} e farli prima aumenta il rischio di infortunio.
\end{ross}

\section{Circuito ad alta sinergia muscolare}
\subsection{Manubri}
Maschi 4/5 Kg mentre le femmine 2/3 Kg. Dopo molto lavoro del deltoide subentra più stress per i muscoli profondi (rotatori).
\begin{enumerate}
\item Step-up con elevazione delle braccia (alzare il ginocchio opposto), alternare la gamba che spinge.
Cercare di dare un condizionamentoe e stabilità generale, senza avere molta similitudine con la
gestualità della tecnica specifica. Si può studiare un circuito che richiami il gesto tecnico per poi
avere un transfer maggiore. \`E utile da alternare a quello generale perché ovvia al problema della
monotonia. L'unico contro è quello di fare movimenti troppo simili della pallavolo (elementi che
presuppongono lo svincolamento del bacino).
\item Affondi diagonali tipo a “v” + imitativo delle braccia in entrata in ricezione. L'affondo
è un movimento che presuppone un adattamento nel gioco. Può essere fatto anche su superfici
instabili (automaticamente si sistemano le posture)
\item Squat + imitativo dello slancio delle braccia a muro (instabilità delle spalle)
\item Affondo laterale e alzata laterale (spalle aperte e scapole vicine). Utilizzare la “respirazione
di forza” e non fisiologica (per bloccare il diaframma) che si usa nelle palestre di fitness.
L'importante è che gli atleti non siano in apnea.
\item Tirate al mento.
\item Affondo avanti ed elevazione avanti con manubri: si lavora sempre su spinta e frenata +
stabilità articolare del ginocchio, mentre le spalle non devono ingobbirsi (come nel bagher). Si crea
anche una costruzione di una base metabolica in quanto nel nostro sport gli scambi lunghi esistono;
\item Squat giù più alzata frontale con bilanciere: le prime volte si chiede che la presa sia comoda
e simmetrica; c'è controllo della zona lombare. Una volta che si domina l'aspetto coordinativo la
grossa discriminante è la velocità esecuzione (quindi importanti accelerazioni). Rapporto tra pesisti
e velocisti.
\item Affondo in avanti con torsione del busto con bilanciere e ritorno. Non meno di 10 –- 12 e
non più di 16. (ca. 20'');
\item Squat giù e lento dietro con bilanciere: veloce la spinta e lenta la discesa.
\item Affondo laterale più bicipide con bilanciere (il bicipite è uno stabilizzatore della spalla), con
controllo della schiena;
\item Spinta dei piedi più pull-over con manubri: presa dei manubri incrociata e spinte verso l'alto
finendo sulla punta dei piedi (gomito abbastanza chiusi);
\item Squat più rematore con manubri.
\end{enumerate}

\subsection{Palla Medica}
\begin{itemize}
\item[-] Discorso preventivo;
\item[-] Accelerare un attrezzo.
\end{itemize}
Si parla prevalentemente di condizionamento dell'attacco.

\begin{defi}\emph{Atletismo}: atletizzare e rendere atleticamente abile un giocatore per il proprio sport.
\end{defi}
Obiettivo è sempre il controllo del tronco.
\begin{enumerate}
\item Caricamento e salto sul posto con slancio della palla medica sopra alla testa.
\item A coppie, presa della palla medica in affondo e uscita dalla posizione porgendo la palla al
compagno.
\item Giavellottista al muro 4 – 5 lanci per ogni piede.
\item Passo – stacco e lancio della palla medica (2 kg);
\item Presa della palla medica da terra, portarla al petto e lancio;
\item Lancio dorsale. (concatenazione della catena posteriore per la spinta in apertura)
\item Presa in salto (1 – 2 kg). Uno lavora e l'altro assiste. 5 -6 a testa e cambio.
\item Imitativo della fast con slancio della palla medica per avanti – alto.
\end{enumerate}


\chapter{Le caratteristiche dei movimenti generali e specifici}

La Pallavolo è uno sport di situazione caratterizzato da movimenti aciclici (che
non si ripetono in modo sistematico nel tempo), non stereotipati (l'ambiente
esterno influisce su di essi) e tattici. \`E uno sport “open skills”, ovvero uno
sport in cui le abilità sono messe in evidenza da un ambiente variabile e poco
predicibile (influenzati da eventi esterni: gli avversari). Nella Pallavolo non
c'è possesso palla.

\begin{libro}
Per disciplina \emph{open skills} si intende una disciplina caratterizzata da
\emph{abilità motorie aperte}, ossia costantemente utilizzate in risposta
alla variabilità situazionale.
\end{libro}

\subsubsection{I Fondamentali}
I vari comportamenti sequenziali che i giocatori attuano durante le azioni
di gioco rappresentano i \emph{fondamentali}, essi sono:
\begin{itemize}
\item[-]Ricezione (azione di cambiopalla);
\item[-]Alzata (azione di cambiopalla);
\item[-]Attacco e copertura (azione di cambiopalla);
\item[-]Servizio (azione di contrattacco);
\item[-]Muro (azione di contrattacco);
\item[-]Difesa (azione di contrattacco);
\item[-]Ricostruzione e attacco con copertura (azione di contrattacco).
\end{itemize}

\subsubsection{Tecniche di Base}
\begin{itemize}
\item[-]Palleggio;
\item[-]Bagher;
\item[-]Battuta float piedi a terra;
\item[-]Schiacciata;
\item[-]Alcune tecniche di spostamento.
\end{itemize}



\section{Metodologia}
Si può classificare l'esercizio (il mezzo) in vario modo:
\begin{itemize}
\item[-]Analitico, sintetico, globale;
\item[-]Ciclico: il medesimo gesto tecnico ripetuto;
\item[-]Guidato: si comunica al giocatore la variazione;
\item[-]Situazionale: si simulano situazioni di gioco (poca prevedibilità).
\end{itemize}


\section{Programmazione}
La programmazione dipende principalmente dal dalla categoria che si deve
allenare. Nel minivolley fino all'under 13 ci si basa sulla motricità e sul
gioco della pallavolo, mettendo le basi. Nell'under 14--18 si lavora
principalmente sulla tecnica.

\section{Prestazione sportiva}
Si basa sulle capacità coordinative, condizionali e di elaborazione tattica.
Queste vengono sviluppate attraverso la metodologia e la programmazione
dell'allenamento sportivo mediante il principio di adattamento.

\subsection{Capacità Coordinative}
Le capacità coordinative, deputate principalmente all'apprendimento, al 
controllo e alla regolazione del movimento, sono le seguenti:
\begin{itemize}
\item[-]\emph{Apprendimento} del movimento;
\item[-]\emph{Controllo} del movimento;
\item[-]\emph{Adattamento} e \emph{trasformazione} del movimento;
\item[-]Tutte le \emph{manifestazioni coordinative secondarie}.
\end{itemize}

\subsection{Capacità Condizionali}
Sono caratteristiche del movimento della pallavolo, fanno riferimento a 
tutti i meccanismi energetici, strutturali, fisiologici e funzionali del
movimento (capacità organico--muscolari), queste sono:
\begin{itemize}
\item[-]\emph{Forza}: definita nel capitolo \ref{fisico};
\item[-]\emph{Rapidità}: la capacità di reagire ad uno stimolo con la massima
velocità possibile;
\item[-]\emph{Resistenza}:  la capacità di mantenere una
 determinata prestazione, esiste quella generale (aerobica, marginale nella
 pallavolo) e quella speciale (resistenza specifica di gara).
\end{itemize}


\section{Complessità dei sistemi}
Controllo delle posture, i processi mentali ed anticipatori analizzati gli
stimoli. Intercettazione (tocco della palla) e cambio direzione e dell'altezza
(variazione di traiettoria della palla). Lateralità (emisfero destro/sinistro).

\subsection{Dominanza}
Un emisfero comanda le dominanze e non sempre c'è un uso simmetrico del corpo.
La \emph{lateralizzazione} comincia ad essere netta dopo i $5$-$6$ anni e
non procede con la stessa velocità su tutti gli organi e le strutture. Ad
esempio i piedi e gli occhi subiscono un ritardo rispetto alla mano.

Risulta importante (nel minivolley) identificare le dominanze di ogni individuo
ed assecondarle, cercare di correggerle può dare disturbi alla lettura, alla
motricità, all'orientamento, etc.

Nella pallavolo ci sono movimenti/fondamentali simmetrici e non, in quelli
simmetrici si hanno meno problemi.

\subsection{Accelerazione e Decelerazione}
Nella pallavolo il concetto di velocità deve essere correlato a quello di
decelerazione. Il sistema accelerazione/decelerazione (enfatizzato
nell'intercettazione della palla) coinvolge altri sistemi: posture statiche e
dinamiche, processi di elaborazione e anticipazione degli stimoli ambientali,
lateralità e modulazione e frequenza del ritmo dei movimenti.

\subsection{Traiettorie}
La conoscenza delle traiettorie avviene per \emph{prove ed errori} e la
misura del processo di apprendimento è la \emph{stabilità} dell'efficacia
motoria, ovvero non ci sono oscillazioni nel numero di esecuzioni corrette e
nel tipo di errore negli esercizi.
Risulta fondamentale imparare a leggere le traiettorie già dal minivolley.

Si evidenziano due fattori condizionanti per la lettura delle traiettorie
ed i movimenti per l'intercettazione: i \emph{posizionamenti} ed il
\emph{concetto di tempo tecnico}.

\begin{libro}
Il \emph{tempo tecnico} è il tempo a disposizione del giocatore nell'esecuzione
di una sequenza percezione--elaborazione--esecuzione.
\end{libro}


\chapter{Introduzione allo Scouting}

Lo scouting è l'azione di rilevamento dei i dati generanti da una partita o da un allenamento.
I dati servono da supporto alla creazione delle statistiche e quindi alle analisi successive, queste confermano o smentiscono le “sensazioni dell’allenatore”, fondate sulla sua
esperienza ed abilità.
L’atleta informato risulta essere più motivato, lo scout aiuta ad eliminare la cultura dell’alibi.

La rilevazione e l'analisi dei dati devono rispondere ai principi dell'oggettività e dell'obiettività.
Il giocatore deve essere valutato per il suo reale rendimento senza condizionamenti emotivi (simpatia o antipatia), senza condizionamenti legati alla spettacolarità di certe azioni.
Lo souting deve essere basato su criteri di valutazione attentamente studiati e codificati dall'allenatore, a cui lo scout-man deve fare fede.

\subsubsection{A cosa serve lo scout?}
Conoscere le caratteristiche dalla proprio squadra;
Conoscere le caratteristiche delle squadre avversarie;
Ottenere dati precisi e misurabili (oggettivi);
Comunicare ed analizzare i dati alla squadra.

\section{Come si effettua lo scouting?}
L'allenatore definisce a priori, per ogni fondamentale, i criteri di valutazione in base alle proprie esigenze, ogni fondamentale può avere un numero variabile di valutazioni.
Lo scouting può essere fatto in tempo reale (durante la partita o allenamento), in qual caso si rende necessaria la figura di scout-man, oppure mediante la registrazione della partita.
Nel primo caso, l'allenatore può usufruire dei dati in tempo reale ed eventualmente intervenire adeguatamente, nel secondo i dati saranno disponibili solo per una seconda analisi.
Di seguito viene definito un sistema basato su 5 criteri di valutazione (simboli o gradi di performance), che risulta tra i più utilizzati. Nulla vieta di variarne i criteri di valutazione o di limitare i fondamentali.

I simboli qui sotto riportati sono i seguenti: $\sharp$ (doppio positivo), $+$ (positivo),
$/$ (slash è un simbolo speciale), $-$ (negativo), $--$ (doppio negativo). A volte viene utilizzato
un altro simbolo speciale ($!$).

\subsubsection{Battuta}
\begin{enumerate}
\item[$\sharp$] Ace, 0, 1 o 2 tocchi dell'avversario;
\item[$+$] Battuta positiva, il pallone ritorna direttamente nel campo della squadra al servizio o la squadra che riceve non può attaccare;
\item[$/$] Battuta “slash”, la squadra che riceve ha un attacco obbligato: palla oltre i 4 metri (gioco obbligato sulle bande) o palla troppo attaccata a rete (gioco obbligato in primo tempo);
\item[$-$] Battuta negativa, la squadra che riceve ha tutte le opzioni d'attacco;
\item[$--$] Errore.
\end{enumerate}

\subsubsection{Ricezione}
\begin{enumerate}
\item[$\sharp$] Ricezione perfetta, il palleggiatore ha la palla entro un raggio di al massimo 1,5 mt ad un altezza adeguata per avere tutte le opzioni d'attacco;
\item[$+$] Ricezione positiva, il palleggiatore ha la palla ad un raggio di poco più di 1/1,5 mt (massimo 2,5) ad un altezza adeguata senza avere il gioco al centro;
\item[$/$] Ricezione “slash”, consente un solo tipo di alzata, oltre i 4 metri o troppo attaccata a rete;
\item[$-$] Ricezione negativa, non consente l'attacco (free-ball);
\item[$--$] Errore, ace subito (0, 1 o 2 tocchi);
\end{enumerate}

\subsubsection{Attacco}
\begin{enumerate}
\item[$\sharp$] Attacco punto;
\item[$+$] Attacco positivo, la squadra che difende non può contrattaccare (compreso il muro rigiocato dalla squadra che ha attaccato);
\item[$/$] Attacco “slash”, la squadra che difende riesce a ricostruire;
\item[$-$] Attacco negativo, la squadra che difende ha tutte le opzioni d'attacco o ha ottenuto il punto a muro;
\item[$--$] Errore diretto;
\end{enumerate}

Spesso si registrano anche le direzioni degli attacchi mediante appositi grafici.

\subsubsection{Muro}
\begin{enumerate}
\item[$\sharp$] Muro punto;
\item[$+$] Muro passivo, favorisce la squadra che difende che può contrattacare agevolmente, oppure torna nel campo della squadra che ha attaccato ma che non può ricostruire un attacco (free-ball);
\item[$/$] Muro ininfluente, la squadra che ha attaccato può rigiocare agevolmente;
\item[$-$] Mani-out;
\item[$--$] Invasione a rete;
\end{enumerate}

\subsubsection{Difesa}
\begin{enumerate}
\item[$\sharp$] Difesa difficile, acrobatica, lontana o con un gesto tecnico evidente;
\item[$+$] Difesa positiva, consente la ricostruzione di un'azione;
\item[$/$] Copertura dell'attacco;
\item[$-$] Difesa negativa, non consente la ricostruzione;
\item[$--$] Difesa non tenuta (tocco errato) o errore di posizione;
\end{enumerate}


\subsection{Formule}
In questa sezione vengono definite le sole tre formule date a lezione e che possono essere richieste durante l'esame, queste riguardano la positività in attacco ed in ricezione e l'efficienza in attacco. Si definiscono per ogni giocatore le seguenti quantità: $A_{\sharp}$ il numero degli attacchi punto, $A_{-}$ il numero degli attacchi negativi, $A_{--}$ il numero degli errori in attacco, $A_{T}$ il numero totale degli attacchi, $R_{\sharp}$ il numero di ricezioni perfette, $R_{+}$ il numero delle ricezioni positive ed $R_{T}$ il numero totale delle ricezioni.

\begin{flushleft}
 \[
\textbf{Efficienza in Attacco:} \quad
 \frac{(A_{\sharp} - A_{-} - A_{--})}{A_{T}} \cdot 100
 \]
\end{flushleft}
 
\begin{flushleft}
\[
\textbf{Positività in Attacco:} \quad
 \frac{A_{\sharp}}{A_{T}} \cdot 100
\]
\end{flushleft}

\begin{center}
\[
\textbf{Positività in Ricezione:} \quad
 \frac{(R_{\sharp} + R_{+})}{R_{T}} \cdot 100
\]
\end{center}


\end{document}