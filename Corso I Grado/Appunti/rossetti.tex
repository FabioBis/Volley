\chapter{Preparazione fisica del settore giovanile}

Le linee guida per la preparazione fisica nel settore giovanile sono le seguenti:
\begin{itemize}
\item Lavoro a \emph{carico naturale} o a carico \emph{minimo};
\item Lavoro in condizioni di sicurezza totale;
\item Continuità del lavoro per tutta la stagione;
\item Utilizzo della tecnica per strutturare alcuni elementi della preparazione atletica;
\item Le modalità del lavoro sono riassumibili attraverso alcuni punti fermi:
\begin{itemize}
\item Il lavoro si apre sempre con una fase di riscaldamento e attivazione muscolo-articolare di circa 10 minuti.
\item Il lavoro si svolge su circuiti a stazioni eseguiti da coppie di atleti per durata di 30 secondi ciascuno (mentre uno lavora l'altro recupera).
\end{itemize}
\item Il lavoro si svolge su 6 stazioni (per 12 atleti);
\item Ogni seduta di preparazione comprende circa 45 minuti dell'allenamento nel periodo di preparazione e circa 30 nel resto dell'anno, tenendo conto che 10/15 minuti vengono sempre utilizzati per il normale riscaldamento e mobilità articolare;
\item In ogni caso è un investimento per evitare che a 16/17 anni l'atleta abbia infortuni e dolori muscolo-articolari.
\end{itemize}

\section{Anatomia della spalla}

\subsection{Cuffia dei rotatori}
I muscoli che compongono la cuffia dei rotatori sono i seguenti:
\begin{itemize}
\item[-]Sopraspinato: abduzione (allontanamento) del braccio e stabilizza;
\item[-]Sottospinato: rotazione esterna del braccio e stabilizza;
\item[-]Piccolo rotondo: rotazione esterna del braccio e stabilizza;
\item[-]Sottoscapolare: adduzione (avvicinamento) del braccio, rotazione interna e stabilizza;
\end{itemize}
Si può aggiungere un muscolo meno noto:
\begin{itemize}
\item[-]grande rotondo: adduzione (avvicinamento) del braccio, rotazione interna.
\end{itemize}

\section{}